\chapter*{Preface to the first edition}


The contents of this book were written over a period of three years, from 2017 to 2020, when I was the Teaching Fellow for the introductory laboratory course at Ashoka University. This course, titled \textit{An Introduction to Physics through Experiments}, was envisioned by the Department as a ``gateway'' course to give students a feel of what they could expect from doing physics at Ashoka. The aim of the course is to help students develop an intuitive understanding of experimental methods, and to develop a familiarity with real-world data.

The laboratory also has a strong focus on improving the student's report-writing skills. A typical iteration generally begins with three introductory sessions during which students are introduced to data collection, graphing, and uncertainty analysis. The sessions usually deal with a conceptually simple experiment that the students are already familiar with, like the Simple Pendulum. After each session, the students are asked to submit a draft report on the experiment by the next week. The course instructor or Teaching Fellow then provides each student with feedback about their drafts (preferably individually) and comments on how they may be improved. The process is repeated three times until the student produces a final report on the introductory experiment after three sessions. The results so far have been extremely promising: after just three sessions, aided by regular feedback, students are often able to write extremely engaging reports and -- what is more important -- are able to do so in their own unique voice.

I would like to express my gratitude to Dr. Sabyasachi Bhattacharya (then C. V. Raman Professor at Ashoka and currently Director of TCG-Crest) for the ideas and years of experience in experimental physics that he brought to the laboratory. I would also like to thank Professor Bikram Phookun who contributed a great deal to the introductory sessions and also helped in editing the entire document and proof-reading it multiple times. This work would not have been possible without his guidance. 

And lastly, I would like to express my appreciation to the first batch of physics students at Ashoka University who took this lab in the Spring of 2018 and who -- despite never having the chance to use these manuals because they weren't yet completed -- contributed greatly to my understanding of all the experiments detailed in this book. I state their names here for posterity: Aditya Singh, Aishwarya Jain, Anand Waghmare, Heer Shah, Nayanika Krishnan, Rahul Menon, Rashmi Gottumukkala, Riya Banerjee, Shwetabh Singh, Sreerag K P, Sreya Dey, Srinidhi Pithani, Vidur Singh, and Yajushi Khurana.




\hfill
\begin{tabular}{@{}l@{}}
Philip Cherian\\
Teaching Fellow (2017 -- 2020)
\end{tabular}
