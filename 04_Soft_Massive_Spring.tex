% \title{The Soft Massive Spring}
% \author{An Introduction to Physics through Experiments}
% \date{}
% \maketitle

\chapter{The Soft Massive Spring}

\section*{Objectives}

\begin{enumerate}
\item To determine the spring constant and the mass correction factor for the given soft massive
spring by static (equilibrium extension) method.
\item To determine the spring constant and the mass correction factor for the given soft massive
spring by dynamic (spring mass oscillations) method.
\item To determine the frequency of oscillations of the spring with one end fixed and the other
end free i.e. zero mass attached.
\item To study the longitudinal stationary waves and to determine the fundamental frequency of
oscillations of the spring with both the ends fixed.
\end{enumerate}

\section*{Apparatus}

\begin{enumerate}[label=\arabic*)]
\itemsep0em
\item A set of soft massive springs
\item A long and heavy retort stand with a clamp at the top end 
\item A set of masses with hooks
\item A signal generator (\textit{Equip-tronics QT-210})
\item A dual output power amplifier with the connecting cords
\item A mechanical vibrator
\item A digital multimeter (\textit{Victor VC97})
\item A digital stopwatch (\textit{Racer})
\item Measuring tapes
\item A set of measuring scales (1.0 m, 0.6 m and 0.3 m)
\end{enumerate}

\section*{Introduction}

A spring is a flexible elastic device which stores potential energy by the straining of the bonds between the atoms of the elastic material of which it is made. A variety of springs are available which are designed and fabricated to suit the various mechanical systems. Some common types of springs are compression springs, extension springs and torsion springs. 

Robert Hooke, a 17th century physicist, studied the behavior of springs under different loads. He established an equation, now known as Hooke’s law of elasticity which states that the amount by which a material body is deformed (the \textit{strain}) is linearly proportional to the force causing the deformation (the \textit{stress}). When applied to a spring, Hooke’s law implies that the restoring force is linearly proportional to the equilibrium extension. In other words,  $$F = - k x,$$ where $F$ is the restoring force exerted by the spring, $x$ is the displacement from the equilibrium position and $k$ is called the spring constant\footnote{The negative sign indicates that the force $F$ is opposite in direction to the extension $x$, hence the term `restoring'.}. For this equation to be valid, $x$ needs to be below the elastic limit of the spring. If $x$ is more than the elastic limit, the spring will exhibit `plastic behavior', where the atomic bonds in the material of the spring get broken or rearranged and the spring no longer returns to its original state. It may be noted that the potential energy $U$ stored in a spring is given by $$U= \frac{1}{2} k x^2$$

Depending on the value of the spring constant, a spring can be called \textit{soft} or \textit{hard}. A spring may also be considered to be massless or massive, depending on the mass that needs to be attached to it to get a considerable extension in the spring. Sometimes springs are also categorized by the ratio of spring constant to the mass of the spring $(\nicefrac{K}{m_s})$. A soft massive spring has a low spring constant and its mass can not be neglected. 

Ideal springs are considered to be massless. Hung vertically without any mass attached, an ideal spring shows no extension. Similarly, with an attached mass $M$ the spring is found to oscillate with a characteristic time period $$T = 2\pi \sqrt{\frac{M}{k}}$$ Practically (for example in the determination of its spring constant), we usually neglect the mass of the spring. However, if the spring extends under its own weight, its mass cannot be neglected and therefore the extension requires a correction. Similarly, they oscillate without any attached mass, implying that the standard formula for the time period of oscillations of a spring needs modification. The corrected formulae have been worked out and, interestingly, one finds that the correction factors $m_\text{corr}$ due to the spring's mass $m_s$ in these two cases are not the same. In this problem, we will experimentally study and verify the modified formulae. 

An extended soft massive spring clamped at both the ends can be assumed to be a uniformly distributed mass system. It has its own natural frequencies of oscillations (corresponding to different normal modes) like a hollow pipe closed at both the ends. Using the method of resonance we will excite and study different normal modes of vibrations of the spring. Here the longitudinal stationary waves will be set up on the extended soft massive spring. 

\section*{Description}

In \textbf{Part A}, we will use the static method, where the equilibrium extension of a given spring will be measured for different attached masses and the spring constant and the mass correction factor will be determined. 

In \textbf{Part B}, we will use the dynamic method, where different masses will be attached to the lower end of the spring with its upper end fixed and corresponding time period of oscillations for such a spring-mass system will be measured. The frequency of oscillations of the spring with the upper end fixed and the lower end free (i.e. with no attached mass) will be determined graphically. 

In \textbf{Part C}, we will use a mechanical vibrator to force oscillations on the spring and excite different normal modes of vibrations of the spring. Thus the longitudinal stationary waves will be set up on the spring. We will measure the frequencies of the excitations corresponding to different normal modes. From these, the fundamental frequency of oscillation with both the ends fixed will be determined. We will compare this frequency with the frequency of oscillations with one end fixed and the other end free as determined earlier in Part B.



\section*{Theory}

\subsection*{Part A}

Let $L_0$ be the length of the spring when the spring is kept horizontal under no tension, $m$ be
the mass attached to the free end of the spring, $L_m$ be the length of the spring when the mass
is attached at its lower end, $S_m$ be the equilibrium extension of the spring for mass $m$, $m_s$ be the mass of the spring, $k$ be the spring constant and $g$ be the acceleration due to gravity.

Thus,

\begin{equation}
S_m = L_m - L_0
\end{equation}

We can determine the expression for $S_m$, by taking extension of a small element and integrating over the total length of the spring,

\begin{equation}
S_m = \left( m + \frac{m_s}{2} \right)\left(\frac{g}{k} \right)
\end{equation}

The factor $(\nicefrac{m_s}{2})$ is known as the mass correction factor in the \textbf{static} case.


\subsection*{Part B}
In case of the soft massive springs, we cannot neglect the mass of the spring since these springs can oscillate without any attached mass. We thus need to modify the earlier expression for $T$. This can be done using the principle of conservation of energy, $$\text{Potential Energy + Kinetic Energy = constant.}$$

The modified expression is found to be

\begin{equation}
T = 2\pi \sqrt{\cfrac{\left(m + \cfrac{m_s}{3}\right)}{k}}
\end{equation}

The factor $(\nicefrac{m_s}{3})$ is known as the mass correction factor in the \textbf{dynamic} case. Note that this factor is different from the mass correction factor in the previous (static) case.

The corresponding frequency is given by $$f_0 = \frac{1}{T}$$


\subsection*{Part C}

As mentioned earlier, the extended spring has its own natural frequencies like a hollow pipe closed at both the ends. Note that both the ends of the spring may be taken to be fixed in the case in question: the upper end is fixed in any case and the amplitude of the lower end is small, as compared to the extended length of the spring, and can be taken to be nearly zero. The natural frequencies correspond to stationary waves; their wavelengths are given by 

\begin{equation*}
\lambda_n = \left( \frac{2}{n}\right) L , \quad \quad n=1,2,3\hdots
\end{equation*}

We know that the speed $v$ of the waves on the spring follow the equation $$v = f_n \lambda_n$$ Thus,

\begin{equation}
\begin{aligned}
f_n = \frac{v}{\lambda_n} &= \frac{v n}{2L}\\
f_1 = \frac{v}{2L}, \quad f_2 = \frac{v}{L},\quad &\hdots,\quad f_n = n f_1
\end{aligned}
\end{equation}

\section*{Experimental Setup}

For Parts A and B, you will need a soft massive spring, a retort stand with a clamp, a set of
masses, a measuring tape or scales and a digital stopwatch. For Part C, you will need a soft massive spring, a long and heavy retort stand with a clamp at the top and a mechanical vibrator clamped near the base of the stand. We will also need a function generator. In this case, the soft massive spring should be clamped at the upper end of the stand. The lower end of the spring should be clamped to the crocodile clip fixed at the centre of the mechanical vibrator. The lower end of the spring will be subjected to an up and down harmonic motion using the mechanical vibrator. It must be ensured that the amplitude of this motion is small enough so that these ends could be considered to be fixed.

\subsection*{Warnings}

\begin{itemize}
\item Do not extend the spring beyond the elastic limit. Choose the value of the maximum mass that may be attached to the lower end of the given spring carefully.
\item Keep the amplitude of oscillations of the spring-mass system just sufficient to get the required number of oscillations.
\item The amplitude of vibrations should be carefully adjusted to the required level using the amplitude knob of the function generator so as to not blow the fuse in the power amplifier. The brighter and more frequently the indicator LEDs flash indicates that the fuse is close to blowing.
\end{itemize}

\section*{Procedural Instructions}

\subsection*{Part A}
\begin{enumerate}
\item Measure the length $L_0$ of the spring keeping it horizontal on a table in an unstretched (all the coils touching each other) position.
\item Hang the spring to the clamp fixed to the top end of the retort stand. The spring extends under its own weight.
\item Take appropriate masses and attach them to the lower end of the spring. 
\item Measure the length $L_m$ of the spring in each case. (For better results you may repeat
each measurement two or three times.) Thus determine the equilibrium extension $S_m$ for each value of mass attached.
\item Plot an appropriate graph and determine the spring constant $k$ of the spring and the mass of the spring $m_s$.
\end{enumerate}

\paragraph{Question 1:} State and justify the selection of variables plotted on the $x$ and $y$ axes. Explain the observed behavior and interpret the $x$ and $y$ intercepts.



\subsection*{Part B}

\begin{enumerate}
\item Keep the spring clamped to the retort stand.

\item Try to set the spring into oscillations without any mass attached, you will observe that
the spring oscillates under the influence of its own weight.

\item Attach different masses to the lower end of the spring and measure the time period of oscillations of the spring mass system for each value of the mass attached. You may measure the time for a number of oscillations to determine the average time period.

\item Perform the necessary data analysis and determine spring constant $k$ and the mass of the spring $m_s$ using the above data. Compare these values to those obtained in the Part A.

\item Also determine frequency $f_0$ for zero mass attached to the spring from the graph.


\paragraph{Question 2:} Does the above method of measuring the total time for a number of oscillations help us to increase the reliability of time period measurement?
\end{enumerate}


\subsection*{Part C}

\begin{enumerate}
\item Keep the spring clamped to the long retort stand.

\item Clamp the lower end of the spring to the crocodile clip attached to the vibrator.

\item Connect the output of the function generator to the input of the mechanical vibrator
through the power amplifier, using the BNC cable.

\item Starting from zero, slowly go on increasing the frequency of vibrations produced by the vibrator by increasing the frequency of the sinusoidal signal generated by the function generator. At one particular frequency you will observe that the midpoint of the spring oscillate with large amplitude indicating an antinode. (You may use a small piece of tissue paper to observe the amplitude at the antinode.) This is the fundamental mode (first harmonic) of oscillation of the spring. Adjust the frequency to get the maximum possible amplitude at the antinode. Measure and record this frequency using the frequency setting on the multimeter.

\item You will find, however, that it is easier to locate nodes (places where the spring does not move) than antinodes. Increase the frequency further and observe higher harmonics identifying them on the basis of the number of loops you can see between the fixed ends. (If you see $n$ nodes -- or fixed points -- between the endpoints, there are $n+1$ loops.)

\item Plot a graph of frequency for different number of loops versus the number of loops. Determine this fundamental frequency $f_1$ from the slope of this graph.

\item Compare this fundamental frequency $f_1$ with the frequency $f_0$ of the spring mass
system with one end fixed and the zero mass attached (as determined in Part B) and
show that $$f_0 = \frac{f_1}{2}$$
\end{enumerate}

\paragraph{Question 3:} Explain why the two frequencies should be related by a factor of two? (You may use the analogy between the spring and an air column.)


\section*{References}

\begin{enumerate}
\itemsep0em
\item J. Christensen, \textit{Am. J. Phys}, 2004, 72(6), 818-828.

\item T. C. Heard, N. D. Newby Jr, Behavior of a Soft Spring, \textit{Am. J. Phys}, 45 (11), 1977,
pp. 1102-1106. 

\item H. C. Pradhan, B. N. Meera, Oscillations of a Spring With Non-negligible Mass, \textit{Physics Education (India)}, 13, 1996, pp. 189-193.

\item B. N. Meera, H. C. Pradhan, Experimental Study of Oscillations of a Spring with Mass Correction, \textit{Physics Education (India)}, 13, 1996, pp. 248-255.

\item Rajesh B. Khaparde, B. N. Meera, H. C. Pradhan, Study of Stationary Longitudinal Oscillations on a Soft Spring, \textit{Physics Education (India)}, 14, 1997, pp. 130-19. 

\item H. J. Pain, \textit{The Physics of Vibrations and Waves}, 2nd Ed, John Wiley \& Sons, Ltd., 1981.

\item D. Halliday, R. Resnick, J. Walker, \textit{Fundamentals of Physics}, 5th Ed, John Wiley \& Sons, Inc., 1997.

\item K. Rama Reddy, S. B. Badami, V. Balasubramanian, \textit{Oscillations and Waves}, University
Press, Hyderabad, 1994.
\end{enumerate}

\newpage