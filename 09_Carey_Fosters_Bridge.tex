\chapter{Carey Foster's Bridge}

\section*{Objectives}

\begin{enumerate}
\item To study the Carey Foster bridge and find the resistance per unit length of its wire. 
\item To find the value of an unknown low resistance using the Carey Foster bridge.
\end{enumerate}

\section*{Apparatus}

\begin{enumerate}[label=\arabic*)]
\itemsep0em
\item A cell
\item Two fixed resistances (1 or 2 $\Omega$)
\item A fractional resistance box
\item A copper strip ($S$)
\item An unknown resistance
\item A galvanometer.
\end{enumerate}

\section*{Introduction}






\section*{Description}

In \textbf{Part A}, you will determine the correction in the actual value of the unknown resistance.

In \textbf{Part B}, you will determine the unknown resistance.

In \textbf{Part C}, 



\section*{Theory}

\subsection*{Part A}


\subsection*{Part B}


\subsection*{Part C}


\begin{question}
\paragraph{Question:} Can we exchange the position of battery and the galvanometer? Which is the preferred arrangement and why?~\\
\paragraph{Question:} Why the procedure involves interchange of R and S?~\\
\paragraph{Question:} Draw the equivalent Wheatstone Bridge.~\\
\paragraph{Question:} How is Carey Foster an improvement over a standard Wheatstone bridge?~\\
\paragraph{Question:} Can the arrangement be used to determine a large unknown resistance? Explain.
\end{question}


\section*{Experimental Setup}





\subsection*{Warnings}
\begin{enumerate}
    \item Clean all contacts thoroughly including the battery terminals and the teeth of the crocodile clips with sand paper.
    \item Make sure that enough wire is exposed while making connections, so that plastic does not come in between.
    \item Check each contact by tugging on the wire and ensuring that it does not come out.
    \item Arrange the elements of the circuit, so that it resembles your circuit diagram.
    \item Make the circuit loop by loop.
    \item There should always be a finite resistance in the circuit, \textit{before} the key is closed.
    \item All balance lengths should be measured from same side and on same scale.
\end{enumerate}


\section*{Procedural Instructions}

{\color{red}Before starting observations, \textit{test} and \textit{debug} your circuit. Some common methods of debugging include checking the battery potential, checking continuity of your circuit using a multimeter. If there is no potential drop across a passive element, then check for continuity.}


\subsection*{Part A}

\begin{enumerate}
    \item Make the connections as shown in the circuit diagram and connect the DC source. (S is a copper strip.)
    \item Take out the zero resistance plug from the resistance box ($R$) and note down the balance length ($l_1$).
    \item Vary the resistance in the resistance box and find the new balance length.
    \item Interchange the position of the copper strip and the resistance box and repeat steps 1 to 3 to determine $l_2$.
    \item Tabulate $l_1 + l_2$.
    \item Plot a graph between $R$ and $l_1 -l_2$ to determine the resistance per unit length.
\end{enumerate}


\subsection*{Part B}

\begin{enumerate}
    \item Make the connections as shown in the circuit diagram and connect the DC source. $S$ is the unknown resistance.
    \item Take out the zero resistance key from the resistance box ($R$) and note down the balance length ($l_1$).
    \item Vary the resistance and find the new balance length.
    \item Interchange the position of the unknown resistance and the resistance box and repeat steps 1 to 3 to determine $l_2$.
    \item Tabulate $l_1 + l_2$.
    \item Plot a graph between $R$ and $l_1 -l_2$ to determine the unknown resistance.
\end{enumerate}

\subsection*{Part C}



\begin{question}
\paragraph{Question:} What should $l_1 + l_2$ be equal to? Explain the discrepancy in your result?~\\
\paragraph{Question:} You observe a finite intercept in Part A. Can you explain why this happens? ~\\ 
\paragraph{Question:} Why is the value of the unknown resistance equal to the separation between the two intercepts?
\end{question}

\section*{References}

\begin{enumerate}
\itemsep0em
\item
\end{enumerate}


\newpage