\chapter{Data Collection, Recording, and Interpretation}

\section{Objectives}

\begin{enumerate}
    \item To understand basic data collection, representation, and interpretation using a simple pendulum.
\end{enumerate}

\section{Introduction}

In this preliminary experiment you will use a simple and familiar system, the simple pendulum, to understand measurement, data collection, and elementary data interpretation. In this experiment, as in many others that you will do in this lab, you will observe a system that is well understood. You will often want to know how a certain property of the system changes when one of the parameters of the system is changed. For example, in this experiment, you might ask: How does the the time period vary with the mass of the bob? How does the time period vary with the length of the pendulum? Alternatively, you may want to set up an experiment that leads to a particular value for a certain property. For example, you may ask: How long does the pendulum need to be for its period to be 1 second?

\section{Data Collection}

Data collection is the observing and recording of data using instruments. You will record data in a \textit{log book} -- preferably a thick and sturdy hard-covered notebook that lasts the entire semester (or, if you intend to major in physics, perhaps all your years at Ashoka).

\subsection{Recording Raw Data}

\subsubsection{Raw Data vs Processed Data}

Raw data is what you see on your measuring instruments. Recording it is the first important step in experimental physics. Raw data is distinct from processed data, and it is important never to forget the difference between them. For example, if you measure the time for 10 oscillations of a pendulum, it is the time for 10 oscillations that is your raw data. You may be interested in the time period of the pendulum (i.e. the time for 1 oscillation) and to do that you will have to divide the measurement you have made by 10; the moment you do this, you have processed your data, i.e. you have performed an operation on it. 
\begin{imp}
In experimental physics, you must never, under any circumstances, convert your data on the fly before recording it: what you must record, first and foremost, is what you see on your measuring instruments. Later, you can process it and change it to any other form that suits you. 
\end{imp}

\subsubsection{Instrument Parameters}

The instrument gives you a number, which will in general have a certain dimension, with units. To make sense of the data you record, you will have to first record the units in which the numbers on the measuring device are expressed. For example, if you use a $mm$ scale to measure the length of an object, you will need to record $mm$ as your units before you take any actual readings. In a certain sense, the unit in which the number on the instrument is expressed is your first observation. (There is a reading that needs to be taken even before that -- the date on which the experiment is being done!)

\subsubsection{Tabulating Data}

In physics we are interested in patterns. One of the most compact and effective ways of recording data so that patterns can be seen easily is the table. For example, we may want to find the answer to the following question: How does the time period of a simple pendulum vary with the mass of the bob? To answer this question, we will want to take many measurements of the time period, one (or more) for each mass of bob. It follows naturally that we should align each mass with the corresponding time period, with one reading following another in vertical progression. Such an arrangement is called a table. Spend some time deciding what data table to draw before beginning your experiment: it will help you decide what data to take.

There is no strict ``right'' way to tabulate data.\footnote{However, there are several wrong ways.} An example is shown in Table (\ref{table:sampledata}).

\begin{table}[!htb]
\centering
\begin{tabular}{|C{4cm}|C{2cm}|C{2cm}|C{2cm}|C{4 cm}|}
\hline
\rowcolor{Gray}
\hline
\textbf{Mass of bob {\color{gray}(g)}} & \multicolumn{3}{|c|}{\textbf{ Time for 10 oscillations {\color{gray}(s)}}} &\textbf{Time Period {\color{gray}(s)}} \\ \hline
{} & Trial 1 & Trial 2 & Trial 3 & {} \\
\hline
{} & {} & {} & {} & {} \\
\hline
{} & {} & {} & {} & {} \\
\hline
{} & {} & {} & {} & {} \\
 \hline
\end{tabular}
\caption{Sample data table}
\label{table:sampledata}
\end{table}

In the table above, the raw data are recorded in the columns for the \textit{Mass of bob} and \textit{Time for 10 oscillations}. Note how the unit in which the measurement is taken (read from the instrument used) is noted next to the title. 

\subsubsection{Data Ethics}

It is essential that none of these readings be changed; if you wish to change a reading, strike it out gently, in such a way that it can still be seen, and write down, next to it or under it, the modified value. It is essential to have a record of all the data you have taken. Remember that this does not need to be excessively neat, only understandable to you and a trained physicist looking at your log book later, e.g. your TF or your instructor.  

If you imagine that the purpose of the experiment is to get a certain result, then you will sometimes be tempted to get rid of ``wrong'' data, but if you remember that the purpose of the experiment is to learn the methods of experimental physics, you will realise that there is no such thing as wrong data.  

This does not imply that you will always use all the data that you record. But, in general, when you discard certain data, or decide not to use it -- a process that is called ``flagging data'' -- you will have a reason for doing so. 

Before you leave the lab show your data log to your TF or course instructor and ask him or her to initial it. 

\subsubsection{``How many readings should I take?''}

This is the question we hear the most often. The answer is of course that \textbf{it depends}. It certainly depends on the experimental setup, and it depends on the different types of uncertainties present in each measurement.

While some scientific measurements are exact,\footnote{Counting the number of parents you have, for example.} others -- such as the sorts of measurements you will be doing in this lab -- are not. When we make a measurement of a quantity, we generally assume that some exact or true value exists for each, based on how we define what is being measured. We attempt to find these quantities as best we can, with the available resources, keeping in mind the what we want to get eventually from the data and the effect that errors in measurement could influence our desired result (more on this in the chapter on error analysis). This understanding helps us to understand how many readings to take. 

You will notice that you sometimes obtain slightly different results on making multiple measurements of the same quantity. We will deal with this in great detail in the chapter on error analysis, but for now you may imagine that these are a result of random fluctuations about the ``true'' value. One way to increase your confidence in experimental data is to repeat the same measurement many times and take the average. (We will have more to say about this later, but let us understand it intuitively first.)

For example, one way to determine the time period ($T$) of a pendulum is to measure the time for 1 oscillation. But we all understand, intuitively, that it is perhaps better to measure the time for 10 oscillations and divide by 10.  This is true. (Why this is true is not, if you think about it, so obvious -- but we will go into that later.)

But we can find the time for 10 oscillations in two different ways: (i) we can find the time for 10 oscillations in a row; (ii) we can find the time for 1 oscillations 10 times. Do these two procedures amount to the same thing? It turns out that they do not -- in the sense that the error associated with the two procedures are different. (Once again, why this is true is not obvious; let us accept it for now and understand it later.) 

Given that these two procedures for measuring $T$ are not identical, one may now ask the following question: If I want to measure the time for 100 oscillations, is it better to do that all at one go or divide it into 10 trials of 10 oscillations each? (The answer to this question is not obvious either!) 

There are all kinds of subtleties in data collection that you will slowly come to grips with. The answer to the question that is the title of this subsection is not obvious, and you will learn it slowly. 

\begin{question}
\paragraph{Question:} Can you explain why taking the time period for a 1000 oscillations may not give you an answer that is much more accurate than 100 oscillations?
\end{question}

\section{The Experiment}

\subsection{Apparatus}

\begin{enumerate}
    \item Metallic bobs of different materials
    \item A length of string
    \item A cork with a slit
    \item A retort stand with an attached protractor
    \item A stopwatch
    \item A scale
    \item A weighing balance
\end{enumerate}

\subsection{Suggested Procedure}

\begin{question} 
\paragraph{Question:} Which are the different physical quantities in this problem that can be varied to potentially change the time period?
\end{question}

\subsubsection{Part A}

In this part of the experiment you will design a simple experiment to determine the variation of the time period with the mass of the bob.

\begin{enumerate}
    \item Begin by deciding which variables you need to fix, and which variables you will change.
    
    \item Draw out an appropriate table in your auxiliary notebooks. Mark out any important details that would help you remember what you've done when you re-read this. Remember to state not only what you have changed, but also what you have kept \textit{fixed}.
    
    \item Decide on the \textbf{number} of readings you will take. When you have arrived at a number, try to \textit{justify} it.
    
    \item Perform the necessary experiment, varying the relevant parameter. Note down your data.
    
\end{enumerate}



\subsubsection{Part B}

In this part of the experiment you will design a simple experiment to determine the variation of the time period with the length of the string from the pivot to the centre of mass of the bob.

\begin{enumerate}
    \item Begin by deciding which variables you need to fix, and which variables you will change.
    
    \item Draw out an appropriate table in your auxiliary notebooks. Mark out any important details that would help you remember what you've done when you re-read this. Remember to state not only what you have changed, but also what you have kept \textit{fixed}.
    
    \item Decide on the \textbf{number} of readings you will take. When you have arrived at a number, try to \textit{justify} it.
    
    \item Perform the necessary experiment, varying the relevant parameter. Note down your data.
    
\end{enumerate}


\subsubsection{Part C}

In this part of the experiment you will design a simple experiment to determine the variation of the time period with the angle of release of the bob.

\begin{enumerate}
    \item Begin by deciding which variables you need to fix, and which variables you will change.
    
    \item Draw out an appropriate table in your auxiliary notebooks. Mark out any important details that would help you remember what you've done when you re-read this. Remember to state not only what you have changed, but also what you have kept \textit{fixed}.
    
    \item Decide on the \textbf{number} of readings you will take. When you have arrived at a number, try to \textit{justify} it.
    
    \item Perform the necessary experiment, varying the relevant parameter. Note down your data.
    
\end{enumerate}


\paragraph{Note:} The repetition is -- of course -- intentional. We have found that students usually jump through these steps and -- as a result -- spend much of their time painstakingly collecting data that is of little or no use. It is essential that you spend some time deciding what exactly you want to collect, and how best you will represent it, before actually spending any time with the apparatus.

\begin{question}
\paragraph{Question:} In each of the above cases, which graph would be the best to plot? Why?~\\

\paragraph{Question:} It ought to be pretty clear that $T$ should depend on the gravity of the Earth. So, ideally, we would like to vary the gravitational force as well, and see how $T$ changes. Can you think of a way in which this can be done?~\\

\paragraph{Question:} On the basis of your experimental results (not what you learnt in school about the simple pendulum) what parameters do you find the time period $T$ depends on? How confident are you of this conclusion? What is the best way to think of the parameter(s) on which $T$ depends?~\\

\paragraph{Question:} What would be the effect -- if any -- of changing the \textbf{shape} of the bob on the time period? Justify your answer.
\end{question}

% \begin{question}

% \end{question}

% \begin{question}

% \end{question}

% \begin{question}

% \end{question} 






\newpage