\title{Diffraction of Light}
\author{An Introduction to Physics through Experiments}
\date{}
\maketitle

\section*{Objectives}

\begin{enumerate}
    \item To determine the wavelength $\lambda$ of the light emitted by a laser source by studying the diffraction of light due to plane diffraction gratings.
    \item To determine the width of the given single slit by studying its diffraction pattern.
    \item To determine the diameter of a given wire by studying its diffraction pattern.
    \item To determine the size of the circular aperture by studying its diffraction pattern.
\end{enumerate}



\section*{Apparatus}

\begin{enumerate}
    \item A 10 mW seminconductor red laser source.
    \item A 5 mW DPSS green laser source.
    \item A set of necessary mounts.
    \item A set of plane diffraction gratings of different grating spacings.
    \item A single helix (spring) set in a holder
    \item A double helix set in a holder
    \item Measuring tapes.
    \item A holder for grating.
    \item A set of screens.  
    \item A single slit of fixed width mounted on a slide.
    \item Two multiple slits mounted on slides.
    \item Circular apertures mounted on slides.
    \item A spirit level
\end{enumerate}

\section*{Introduction}
	 
In Opticks (1704), Issac Newton wrote, ``Light is never known to follow crooked passages nor to bend into the shadow''. He explained this by describing how particles of light always travel in a straight line, and how objects kept in the path of the light cast a shadow because the particles can never spread out behind the object. However, a set of experiments on the propagation of light through small apertures performed by Francesco Grimaldi, Augustine Fresnel, Thomas Young and a few others firmly established that light actually enters into the shadow region with a definite pattern when it passes through around an edge. The resulting pattern depends on the relative size of the aperture or obstacle and the wavelength of light. If the size is much larger than the wavelength, the bending will be almost unnoticeable. However, if the two are similar in size, the diffraction will be considerable.    
     
In this experimental problem, we will use a low power solid-state laser as a source of an intense beam of monochromatic light. When light from a distant source (or a laser source) passes around a thin aperture or through a narrow aperture and is then intercepted by a viewing screen, the light produces a pattern on the screen called a \textit{diffraction pattern}. When such a beam is incident on various diffracting components like a plane diffraction grating, a single slit, a wire mesh or a two-dimensional diffraction grating, the light emerging from these components show a variety of interesting diffraction patterns. This pattern consists regions of maximum and minimum intensities, which characterise the diffracting object. 

\section*{Theory}

\subsection*{Plane Diffraction Grating}

A transmission diffraction grating consists of a large number of slits separated from one another by an opaque region. The grating concentrates the diffracted light along a particular direction in contrast to the single slit, which has a rather broad diffraction maximum. The maxima (bright intense spots) produced by a grating are usually called the principal maxima. They are quite intense and are also widely separated; what cannot be detected visually are the large number of secondary maxima which lie between neighboring principal maxima.

The expression, relating wavelength $\lambda$ of light used and the grating spacing $d$, with angle of deviation $\theta$ is,    
\begin{equation*}
    d \sin{\theta_m} = m \lambda,  \quad\quad\quad \text{for    m  = 1, 2, 3,}\hdots
\end{equation*}

In the above expression, $m$ represents the order of \textbf{maxima} points and the angle $\theta_m$  corresponds to  $m$th  order maximum intensity point. This relation is valid for a single slit and for the wire like obstacle.

Fig. 1

\subsection*{Single Slit}

When a (monochromatic) beam of light such as a laser is incident on a narrow single slit, the light emerging from the slit shows a diffraction pattern on a screen. The distribution of the intensity of light received on a screen show a pattern of varying intensity consisting of a bright central maximum with alternate minima and maxima of decreasing intensity on either side, known as a \textit{Fraunhofer diffraction pattern}.

The positions of the \textbf{minima} (zeros) of the intensity distribution pattern of a narrow slit due to a plane wave front (the Fraunhofer diffraction pattern) are given by the relation 

\begin{equation*}
    a \sin{\theta_m} = m \lambda  \quad\quad\quad \text{for    m  = 1, 2, 3,}\hdots,
\end{equation*}

where $\lambda$ is the wavelength of the incident light, $a$ is the width of the slit and $\theta_m$ is the angle corresponding to $m$th minimum. 

Fig. 2



\subsection*{Circular Aperture}

The diffraction pattern due to a circular aperture (known as an \textit{Airy diffraction pattern}) is similar to a single slit diffraction but the mathematics involved is more complicated which gives the expression nearly identical to that of the single slit. Hence we may apply the same expression to the diffraction due to a circular aperture, 

\begin{equation*}
    d \sin{\theta_m} = \overline{m} \lambda,  \quad\quad\quad \text{for    m  = 1, 2, 3,}\hdots
\end{equation*}

where $d$ is the diameter of the circular aperture and $\theta_m$ is the angle of deviation for the $m$th dark ring. The variable $\overline{m}$ has the following values:

\begin{equation*}
    \begin{aligned}
        m = 1 &\quad& \overline{m}=1.22
        m = 2 &\quad& \overline{m}=2.23
        m = 3 &\quad& \overline{m}=3.23
        m = 4 &\quad& \overline{m}=4.24
    \end{aligned}
\end{equation*}

\subsection*{Cylindrical wires}

 
 Fig. 3(a) Single cylindrical wire                                Fig. 3(b) Two wires

A laser beam of wavelength $\lambda$, falling normally on a cylindrical wire of diameter $a$ is diffracted in the direction perpendicular to the wire. The resulting intensity pattern as observed on a screen is shown in \ref{Fig. 3(a)}. It will not have escaped your attention that this pattern is very similar to that observed for a single slit. This is due to Babinet’s principle, which states that complementary objects produce the same diffraction pattern. The mathematics of this are far beyond the scope of this lab session.

The intensity distribution as a function of angle $\theta$ with the incident direction is given by

\begin{equation*}
    I(\theta) = I(0) \left( \frac{\sin \beta}{\beta} \right)^2, \quad \quad  \beta = \frac{\pi a \sin \theta}{\lambda}
\end{equation*}


The central spot ($\beta = 0$ and $\sin\beta = 0$) is bright. For other angles, when $\sin\beta=0$ but $\beta \neq 0$, the intensity vanishes.

Thus the intensity distribution has $n$th minimum at the angle $\theta_n$ given by

\begin{equation*}
    a \sin\theta_m = \pm m \lambda
\end{equation*}

Here the $\pm$ refers to either side of the central spot ($\theta = 0$).

The diffraction pattern due to two parallel identical wires of width $a$ kept at a distance $d$ from each other (Fig. 3(b)) is a combination of two patterns (diffraction due to a single wire and interference due to two wires).

The resultant intensity distribution is given by   

\begin{equation}
    I(\theta) = I(0) \cos^2\delta \left( \frac{\sin \beta}{\beta} \right)^2, \quad \quad  \beta = \frac{\pi a \sin \theta}{\lambda}, \delta = \frac{\pi d \sin \theta}{\lambda}
\end{equation}

For a screen placed at a large distance $D$ from the wire, the positions of the minima on the screen are observed at 

\begin{equation*}
    \begin{aligned}
        x_{\pm n} = \pm n \frac{\lambda D}{a}, &\quad& \text{due to diffraction},\\
        x_{\pm m} = \pm \left( m - \frac{1}{2}\right) \frac{\lambda D}{d}, &\quad& \text{due to interference}
    \end{aligned}
\end{equation*}


\subsection*{Single and double helices:}
  
               
Fig. 4 (a) A single helix Fig. 4 (b) A double helix

Now consider a set of four identical wires, the net intensity distribution is a combination of diffraction from each wire and interference due to pairs of wires and hence depends on $a$, $d$ and $s$ \ref{Figure 5 (a)}.

In other words, the combination of three different intensity patterns is observed.

  

Fig. 5 (a) Projection of a double helix    Fig. 5(b) Double helix given in the sample


\section*{Procedural Instructions}

\subsection*{Part A}

Observe effect of colour and also of white light on the diffraction pattern obtained by a suitable grating. Then choose an appropriate diffraction grating and perform the measurements to determine the wavelength $\lambda$ of the laser. 

\paragraph{Question:} Estimate the error in the value of the wavelength of light.

\paragraph{Question:} What are the sources of error in the above-determined value of $\lambda$ ? What measures should be taken to minimize these errors? 

Tilt the grating at an angle and see how this affects the diffraction pattern.

\subsection*{Part B}

Repeat Part A, for a different gratings (with different values of d), and calculate wavelength $\lambda$ as accurately as possible.

\subsection*{Part C}

Design and perform the necessary experiment with a single slit of fixed width and determine the width d of the given single slit.

Tilt the slit at an angle and see how this affects the diffraction pattern.

\subsection*{Part D}

Now take the given circular aperture as the diffracting object and determine the diameter of the circular aperture.

\subsection*{Part E}

Study of the diffraction pattern due to a helical spring and determine pitch of the spring and thickness of its wire. 

\paragraph{Question:} How does this part of the experiment relate to Part C? Can you explain the form of the diffraction pattern observed?

\subsection*{Part F}

Study of the diffraction pattern due to a double helix (as in our DNA) and determine all its parameters \ref{Fig 5 (b)}


\paragraph{Question} Explain how will be the diffraction pattern, observed using a laser source with
\begin{enumerate}[label=(\alph*)]
    \itemsep0em
    \item A fine wire mesh,
    \item A square aperture,
    \item A rectangular aperture.
\end{enumerate}




\subsection*{Precautions}

\begin{enumerate}
    \item Never look directly at a laser beam with the naked eye or any optical device. It may damage the eye permanently.
    \item Never allow a laser beam to shine in the area of anyone’s eye, as it is harmful to eyes.
    \item Never place highly reflective objects (such as rings, watches, and glassware) in the path of the laser beam.
    \item For proper working of laser, it should be kept ON throughout. Do not put it OFF until you complete all the readings, but if you do not need the laser beam for measurements or alignment, use the light-blocking screen to block the Laser beam.
    \item Treat the laser source as any other electrical device. It should never be tampered with, while the power cord is connected.
    
\end{enumerate}


\section*{References}

\begin{enumerate}
\itemsep0em
    \item Eric Stanley, Am. J. Phys., Vol.- 54, No.-10, October 1986, pp. 952.
    \item F.A. Jenkins, H.E. White, \textit{Fundamentals of Optics}, Third Edition, Mc-Graw Hill Kogakusha Ltd., Toyko, Japan, 1957, pp. 288-309, 328-350.
    \item F.W. Sears, \textit{Optics}, Third Edition, Asia Publishing House, 1958, pp 221-252.
    \item R.W. Ditchburn, \textit{Light}, Second Edition, The English Language Book Society and Blackie \& Son Ltd., 1963, pp 162-237.
    \item John Beynon, \textit{Introductory University Optics}, Prentice-Hall of India Pvt. Ltd., New Delhi (India), 1998, pp 158-190.
    \item Rajpal S. Sirohi, \textit{Wave Optics and its Applications}, Orient Longman Limited, (India), 1993, pp 169-210.
\end{enumerate}