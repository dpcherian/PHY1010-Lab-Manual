\documentclass[a4paper,10pt,oneside]{book}
\usepackage{titling}
% \pretitle{\begin{center}\huge}
% \posttitle{\par\end{center}\vspace{\baselineskip}}
% \preauthor{\normalfont\normalsize\begin{center}\begin{tabular}[t]{c}}
% \postauthor{\end{tabular}\end{center}\vspace{\baselineskip}}
\usepackage[utf8]{inputenc}
\usepackage{amsmath}
\usepackage{graphicx}
\usepackage{tabularx}
\usepackage{subcaption}
\usepackage[]{caption}

\usepackage[a4paper, total={6in, 8in}]{geometry}
\usepackage{tcolorbox}
\usepackage{hyperref}
\usepackage{physics}

\usepackage{nicefrac}
\usepackage{enumitem}
\graphicspath{{figs/}}

\setlength{\parskip}{\baselineskip}%
\setlength\parindent{0pt}

\usepackage{color, colortbl}
\usepackage{bm}
\definecolor{Gray}{gray}{0.9}

\usepackage{arydshln}

\newcolumntype{L}[1]{>{\raggedright\arraybackslash}p{#1}}
\newcolumntype{C}[1]{>{\centering\arraybackslash}p{#1}}
\newcolumntype{R}[1]{>{\raggedleft\arraybackslash}p{#1}}



% \newtcolorbox{imp}{colback=orange!5!white,colframe=orange!75!black,fonttitle=\bfseries,title=Important}

% \newtcolorbox{question}{colback=red!5!white,colframe=red!75!black,fonttitle=\bfseries,title=Question}

\newtcolorbox{imp}{colback=orange!5!white,colframe=orange!75!black,fonttitle=\bfseries}

\newtcolorbox{question}{colback=red!5!white,colframe=red!75!black,fonttitle=\bfseries}

\newtcolorbox{tip}{colback=green!5!white,colframe=green!35!black,fonttitle=\bfseries,title=Tip}



%% [[ DEFINING `POLYTECHNIQUE' COLOURS
\definecolor{Blue}{RGB}{0,59,92}
\definecolor{Grey}{RGB}{74,73,72}
\definecolor{Red}{RGB}{247,38,22}
\definecolor{DarkRed}{RGB}{197,26,27}
\definecolor{Yellow}{RGB}{223,176,0}
\definecolor{White}{RGB}{255,255,255}
%% END POLYTECHNIQUE COLORS]]


\usepackage{color,soul}
\usepackage{tikz}
\setulcolor{Red}
\setul{}{2pt}

\usepackage{chngcntr} 
\counterwithin{section}{chapter}
\makeatletter
\@addtoreset{chapter}{part}
\makeatother 
\renewcommand{\thepart}{\Roman{part}}



\begin{document}

\title{\vspace{8cm}\huge \textcolor{Blue}{\textsc{PHY102:\\An Introduction to Physics through Experiments}}}
\date{}

\vbox{
\maketitle
\tikz[remember picture,overlay]\node[shift={(-1,1)},opacity=0.6] at (current page.south east) {\includegraphics[width=17.5cm]{logo}};
}

\thispagestyle{empty}
\newpage

\tableofcontents

\vbox{
\textcolor{Blue}{\part{Introductory Sessions}}
\tikz[remember picture,overlay]\node[shift={(-1,1)},opacity=0.6] at (current page.south east) {\includegraphics[width=17.5cm]{logo}};
}

\renewcommand{\chaptername}{Session}

\title{Data Collection: The Simple Pendulum}
\author{An Introduction to Physics through Experiments}
\date{}

\maketitle

\section{Objectives}

\begin{enumerate}
    \item To understand basic data collection, representation, and interpretation.
    \item To observe the variation of the time period of a simple pendulum with different parameters.
\end{enumerate}

\section{Introduction}

In this preliminary experiment you will be exposed to the ideas of measurement, data collection, and elementary data interpretation using the simple pendulum. 

\subsection{Data Collection}

\subsubsection{Raw-Data Tables}

In this laboratory you will be observing physical phenomena that are relatively well understood. In general, you will be asked to either verify certain formulae or to extract the value of certain physical quantities from them. As a result, you would usually have an \textbf{independent variable} that you will change over the course of collecting data, and a \textbf{dependent variable} which you will \textit{measure}. The data that you will take will have to be taken systematically, in order for you to make full use of it. As a result, it is imperative that you learn to tabulate your raw data well, so that you do not run into trouble later.

You will need a lab (auxiliary) notebook within which you will take your readings. It is essential that none of these readings be changed; if you have noted down something wrong, strike it out, and write down the correct value. However, it is essential to have a record of all the data you have taken. Remember: this does not need to be neat, only understandable to \textbf{you}. The TA or course instructor will have to initial your data. Do not underestimate the temptation to change your data: never discount points unless you have adequate reason to do so, and \textbf{never} change the value of a reading so that it fits a trend that you imagine is the `right' one.

There is no strict `right' way to tabulate data. However, there are several wrong ways. Here is a simple example: 

\begin{table}[!htb]
\centering
\begin{tabular}{|C{4cm}|C{2cm}|C{2cm}|C{2cm}|C{4cm}|}
\hline
\rowcolor{Gray}
\textbf{Independent Variable {\color{gray}(unit)}} & \multicolumn{3}{|c|}{\textbf{Dependent Variable {\color{gray}(unit)}}} &\textbf{Derived Quantity {\color{gray}(unit)}} \\ \hline
{} & Trial 1 & Trial 2 & Trial 3 & {} \\
\hline
{} & {} & {} & {} & {} \\
\hline
{} & {} & {} & {} & {} \\
\hline
{} & {} & {} & {} & {} \\
 \hline
\end{tabular}
\caption{Sample data table}
\label{sampledata}
\end{table}

Spend some time deciding what data table to draw before beginning your experiment: it will help you decide what data to take.

\subsubsection{``How many readings should I take?''}

This is the question we hear the most often. The answer is of course that \textbf{it depends}. It certainly depends on the experimental setup, and it depends on the different types of uncertainties present in each measurement.

While some scientific measurements are exact\footnote{Counting the number of parents you have, for example.}, others -- such as the sorts of measurements you will be doing in this lab -- are not. When we make a measurement, we generally assume that some exact or true value exists based on how we define what is being measured. We attempt to find this quantity as best we can, with the available resources. 

You will notice that you obtain slightly different results on making multiple measurements of the same quantity. We will deal with this in great detail in the {\color{red}Error Analysis} section, but for now you may imagine that these are a result of random fluctuations about the `true' value. One way to increase your confidence in experimental data is to repeat the same measurement many times. If the errors are truly random, then there should be just as many \textit{above} the true value as \textit{below}. Thus, when you average your answer, you should get a more accurate answer.

For example, one way to estimate the amount of time it takes something to happen is to simply time it once with a stopwatch. However, this would involve random errors due to -- among other things -- your reaction time. You can decrease the uncertainty in this estimate by making this same measurement multiple times and taking the average. The more measurements you take, the better your estimate will be\footnote{Provided there is no problem with the clock! We will speak in detail of such systematic errors later.}.

Thus, before you decide how many readings you wish to take, you need to decide \textbf{what uncertainty it is you are trying to eliminate}. Here is a simple example of how to think about it in the experiment at hand, timing a simple pendulum:

\begin{enumerate}
    \item Let us assume the predominant uncertainty in the time period comes from human error: i.e., the fraction of time between the pendulum beginning an oscillation, and you starting the stopwatch.
    
    \item It turns out that this is roughly 0.3 seconds\footnote{Think of a way to estimate this.}. In other words, timing a pendulum that takes one second to oscillate would have an uncertainty of roughly 0.6 seconds\footnote{0.3s when you start the stopwatch, and 0.3s when you stop it.}! 
    
    \item You would like to reduce the effect this uncertainty has on the measurement, and so \textbf{one} oscillation is certainly not enough. What about 10? The total time of measurement will now be 10 seconds. However, the total uncertainty will \textit{still} be 0.6 seconds, since you're still only starting and stopping the stopwatch one time! 
    
    \item To find the time period, you would just divide by the number of oscillations (a fixed number that you know with absolute certainty), and this will cause the error to be divided by that number as well! Thus, you have an estimate of the time period, to within 0.06 seconds!
\end{enumerate}

\begin{question}
\paragraph{Question:} Since this is true, can you explain why taking the time period for a 1000 oscillations may not give you an answer that is much more accurate than 100 oscillations? ~\\

\paragraph{Question:} If you don't know what the predominant source of uncertainty is, how could you begin to estimate how uncertain your readings are?
\end{question}

\subsection{Data Representation}

Your performance in the lab will be judged almost exclusively through the reports of the work that you have done. You may have exhibited great ingenuity while collecting data or performing an experiment, but if that is not reflected in your final submission for the experiment, there is no way for us to grade you on it.

\subsubsection{Report}

Your lab report is as much an opportunity to show the instructor what you have learnt as it is to practice for composing professional technical reports after your graduation. Here are some things to keep in mind that will help you out when your report is graded.

Make your lab report \textbf{clear} and \textbf{easy to follow}. A common mistake students make is to write extremely long and elaborate reports that convey very little. We are not looking for pages and pages of writing; in fact, from experience, we have seen that the best reports are usually \textbf{short}. Just make sure that someone who hasn't done the lab can understand what you did and how you did it. It is your job to decide what is relevant and what isn't. This is good training for the future. 

\begin{imp}
Your lab report may be written by hand, on Microsoft Word, or using \LaTeX. Of these options, we suggest you begin using \LaTeX\, as soon as possible: it makes your work look professional, and learning to use it is good practice.
\end{imp}

\textbf{Data does not lie.} Contrary to popular opinion, grades are not assigned on how close the value obtained agrees with the ``correct'' one. If you are supposed to verify a law of nature, and you end up disproving it, that's fine, provided that you say that you disprove it. If however you cannot verify it, but say that you can, we can only assume that you didn't understand the experiment. If your results are completely different from established values, then you have probably measured or calculated something incorrectly. 

\begin{imp}
The lab report should contain \textbf{all} the information required for someone else to reproduce your experiment \textbf{and} its results, especially if your value deviates significantly from the agreed value; we should be able to trace back exactly what you did wrong.~\\

\textbf{A wrong answer with a clear path as to how you got there is worth significantly more than a right answer that appears unjustified or out of nowhere. }
\end{imp}

The handouts contain questions about the experiment are scattered around the manual. Make sure you answer them as you write your report; these questions are included at critical points to check your comprehension, and the answers often provide ideas of what to include in a good report. 

\begin{imp}
Spend some time thinking about the format of your lab report. We don't expect something publishable in Physical Review Letters, but on the other hand, we don't expect four pages of stream-of-consciousness writing either.
\end{imp}

\subsubsection{Figures, Graphs, and Tables}

Figures, Graphs, and Tables have a required standard of presentation, usually much higher than those you might put in your auxiliary book. If possible, have any tables and figures at the top and/or bottom of a page, and do not wrap text around figures or tables.

\begin{imp}
In general, you may use any software for data analysis. Beginners will usually prefer to use Microsoft Excel or Google Sheets. However, as you progress, it is \textbf{very strongly} advised that you use this lab to learn how to do very elementary plotting and data analysis using Python and Jupyter Notebook.
\end{imp}

\paragraph{Figures:} 
\begin{enumerate}
    \item The figures you will include in your lab report will usually be descriptions of the apparatus or circuit diagrams. 
    \item In general, they should be well marked and labelled and you should be able to refer to them through the report to better explain what you've done. 
    \item All figures should have a label (such as `Figure 1:' or `Fig.1:') at the start of a caption explaining what they describe. 
    \item Captions should be short and sufficiently informative so that anyone with some knowledge of the experiment will understand what it represents without having to refer to your report.
\end{enumerate}

\paragraph{Graphs:}
\begin{enumerate}
    \item Your graphs should be easy to read and clearly show all key features.
    \item Graphs are figures, and should therefore have a figure number at the start of their caption (`Fig.1:', not `Graph 1:').
    \item Consider whether various results can be combined into a single graph.
    \item Here are some things your graph \textbf{must} have:
    
    \begin{enumerate}
        \item Labelled axes with units,
        \item Error bars, indicating the uncertainty with which you know the location of the point,
        \item A white background,
        \item Sensible maximum and minimum values so that most of the space is filled by the graph.
    \end{enumerate}
    
    \item Your graph \textbf{does not} need to have:
    
    \begin{enumerate}
        \item A title: all the information about the graph should be in the caption,
        \item Grid-marks or a border,
        \item Legends (unless the graph cannot be understood without one): this information would preferably also go into the caption.
    \end{enumerate}

\end{enumerate}


\paragraph{Tables:}

\begin{enumerate}
    \item Like figures, Tables should be labelled and have a caption. However, they are \textbf{not} figures, but should instead be labelled `Table 1:', etc.
    \item All the entries, including the headings, should fit comfortably in the width or height of the columns or rows; long headings should be avoided. 
    \item The headings of the columns should include the \textbf{units} of the measurement. The readings themselves should not have units.
    \item Include the uncertainties in every reading.
    \item Make sure the rows and columns are the same size throughout the table.
\end{enumerate}

\begin{imp}
Don’t include tables of data when the information is adequately given in a graph or by a few words of text; this is redundant and wasteful of space. 
\end{imp}

\subsection{Data Interpretation}

The last step in your report is an \textbf{analysis} of your data. Usually, this is something that you can get from your graphs. Throughout this course you will be asked to plot graphs and to extract the relevant information from them. In general, this is something that is better learnt by doing rather than by reading.  Once a regression line has been found, the equation must be interpreted in terms of the context of the situation being analysed.

In general, it is good practice to always plot \textbf{linear} graphs whenever possible. In certain cases, the relationships between physical quantities are linear, and this is easy: for example, if a ball is dropped from rest from some height, its \textbf{velocity} $v$ varies as

\begin{equation*}
    v = g t
\end{equation*}

Plotting $v$ against $t$ will give you a \textbf{linear} graph whose slope is the acceleration due to gravity. However, such a variation is not always guaranteed for all physical quantities. For example, the \textbf{position} $S$ of the same ball varies as

\begin{equation*}
    S = \frac{1}{2} g t^2
\end{equation*}

which is a \textit{quadratic} relationship. In some cases, you might be tempted to plot $S$ as a function of $t$ and fit a quadratic curve. In general, it is better to plot a graph between $S$ and $t^2$. These two quantities exhibit a linear relationship which is easier to fit, and whose slope gives you $g/2$.

One reason for this is because when you have a very small number of points, different polynomials may appear to fit it equally well.


\section{Theory: The Simple Pendulum}

The simple pendulum is a point mass suspended from a string of negligible mass attached to a pivot point, as shown in Figure (\ref{simple}). 

\begin{figure}[!htb]
    \centering
    \includegraphics[scale=0.5]{figs/simplependulum.png}
    \caption{The Simple Pendulum}
    \label{simple}
\end{figure}

If a pendulum is set in motion so that is swings back and forth, its motion will be periodic. The time that it takes to make one complete oscillation is defined as the \textbf{time period} $T$ of the pendulum. Most of you will probably recognise the following formula for $T$: 

\begin{equation}
    T = 2 \pi \sqrt{\frac{l}{g}} 
    \label{TimeP}
\end{equation}

where $l$ is the length of the pendulum from its fixed point, and $g$ is the acceleration due to gravity. Not so many of you, however, will realise that this is only true for \textbf{small displacements} around the equilibrium position. In general, when the pendulum is displaced from its equilibrium position, it experiences a restoring force $m g \sin\theta$. The differential equation describing its motion can be obtained from Newton's Second Law  $m\vb{a} = \vb{F}$.

\begin{equation}
    m l \dv[2]{\theta}{t} = - m g \sin\theta
\end{equation}

This is a difficult differential equation to solve. In the approximation that the angle $\theta$ is very small, we can replace $\sin\theta \approx \theta$, and we are left with the following differential equation:

\begin{equation}
    \dv[2]{\theta}{t} + \frac{g}{l} \theta = 0
\end{equation}

It is in this approximation that the simple pendulum is a \textbf{simple harmonic oscillator}, a very important model that you will not stop seeing the end of. In general, a quantity $Q$ is considered observe simple harmonic variation with respect to some parameter $t$ (not necessarily time) if it satisfies the following differential equation:

\begin{equation*}
    \dv[2]{Q}{t} + \omega_0^2 Q = 0
\end{equation*}

where $\omega_0$ is the time period of the oscillation, and is related to the time period of the oscillation through 

\begin{equation*}
    T = \frac{2\pi}{\omega_0}
\end{equation*}

Comparing, you should see that in our case, $$\omega_0 = \sqrt{\frac{g}{l}}$$ and $$T = 2\pi \sqrt{\frac{l}{g}}$$

Thus, it should be clear that it is only when the amplitude is \textbf{small} that the time period follows this equation, since it is only then when you can approximate $\sin\theta$ with $\theta$.

In this introductory experiment you will begin by verifying the above formula for the time period for small angles. You will then attempt to explore any variation of the time period with angle.

\section{Apparatus}

\begin{enumerate}
    \item Metallic bobs of different materials
    \item A length of string
    \item A cork with a slit
    \item A retort stand with attached protractor
\end{enumerate}

\section{Suggested Procedure}

\begin{question} 
\paragraph{Question:} Which are the different physical quantities in this problem that can be varied to potentially change the time period\footnote{\textbf{Warning!} Do not use Equation (\ref{TimeP}) to answer this: we've already seen that that only works under certain special cases}?
\end{question}

\subsection{Part 1}

In this part of the experiment you will design a simple experiment to determine the variation of the time period with the mass of the bob.

\begin{enumerate}
    \item Begin by deciding which variables you need to fix, and which variables you will change.
    
    \item Draw out an appropriate table in your auxiliary notebooks. Mark out any important details that would help you remember what you've done when you re-read this. Remember to state not only what you have changed, but also what you have kept \textit{fixed}.
    
    \item Decide on the \textbf{number} of readings you will take. When you have arrived at a number, try to \textit{justify} it.
    
    \item Perform the necessary experiment, varying the relevant parameter. Note down your data.
    
    \item Plot an appropriate graph that accurately depicts your results.
\end{enumerate}

\begin{question}
\paragraph{Question:} What graph would be the best to plot? Why? ~\\

\paragraph{Question:} What is your conclusion? How confident are you of this conclusion?
\end{question}


\subsection{Part 2}

In this part of the experiment you will design a simple experiment to determine the variation of the time period with the length of the string from the pivot to the centre of mass of the bob.

\begin{enumerate}
    \item Begin by deciding which variables you need to fix, and which variables you will change.
    
    \item Draw out an appropriate table in your auxiliary notebooks. Mark out any important details that would help you remember what you've done when you re-read this. Remember to state not only what you have changed, but also what you have kept \textit{fixed}.
    
    \item Decide on the \textbf{number} of readings you will take. When you have arrived at a number, try to \textit{justify} it.
    
    \item Perform the necessary experiment, varying the relevant parameter. Note down your data.
    
    \item Plot an appropriate graph that accurately depicts your results.
\end{enumerate}


\begin{question}
\paragraph{Question:} What graph would be the best to plot? Why? ~\\

\paragraph{Question:} What is your conclusion? How confident are you of this conclusion?
\end{question}

\subsection{Part 3}

In this part of the experiment you will design a simple experiment to determine the variation of the time period with the angle of release of the bob.

\begin{enumerate}
    \item Begin by deciding which variables you need to fix, and which variables you will change.
    
    \item Draw out an appropriate table in your auxiliary notebooks. Mark out any important details that would help you remember what you've done when you re-read this. Remember to state not only what you have changed, but also what you have kept \textit{fixed}.
    
    \item Decide on the \textbf{number} of readings you will take. When you have arrived at a number, try to \textit{justify} it.
    
    \item Perform the necessary experiment, varying the relevant parameter. Note down your data.
    
    \item Plot an appropriate graph that accurately depicts your results.
\end{enumerate}


\begin{question}
\paragraph{Question:} What graph would be the best to plot? Why? ~\\

\paragraph{Question:} What is your conclusion? How confident are you of this conclusion?
\end{question}

\paragraph{Note:} The repetition is -- of course -- intentional. We have found that students usually jump through these steps and -- as a result -- spend much of their time painstakingly collecting data that is of little or no use. It is essential that you spend some time deciding what exactly you want to collect, and how best you will represent it, before actually spending any time with the apparatus.



\section{Questions}

\begin{question}
\paragraph{Question:} What would be the effect -- if any -- of changing the \textbf{shape} of the bob on the time period? Justify your answer.
\end{question}
% \title{Measurement: Errors and Error Analysis}
% \author{An Introduction to Physics through Experiments}
% \date{}
% \maketitle

\chapter{Measurement: Errors and Error Analysis}
\section{Objectives}

\begin{enumerate}
    \item To learn to use the Vernier Calliper and Screw Gauge.
    \item To study the propagation of errors from measured to derived quantities.
\end{enumerate}


\section{Theory: Measurement}

\subsection{The Vernier Calliper}

You will regularly come across vernier scales in the lab when using callipers, screw-gauges, angular vernier scales (in spectrometers), and travelling microscopes. Vernier scales allow you to read off a value more precisely than when using an ordinary scale. In this section, we will explain how this works.

First consider a `main scale' with a least count of 1 unit (you could imagine this is $1mm$, if you wish). This implies that any distance between, say, the 2 and 3 unit marks cannot be determined accurately. In other words, the best you could say is that an object is ``2 and a bit'' units. The vernier scale allows you to \textbf{quantify} this `bit'.  

The method is rather ingenious: the vernier scale has a certain number of divisions (say 10, for the purpose of our example) that are equally spaced (see Figure (\ref{fig:vernier_1})). These 10 divisions are set to coincide with 9 divisions of the main scale.

\begin{question}
    \paragraph{Question:} What is the spacing between two vernier scale divisions? Is it:
    \begin{enumerate}
        \item 1 unit?
        \item 0.1 units?
        \item 0.9 units?
    \end{enumerate}
\end{question}



\begin{figure}[!htb]
    \centering
    \includegraphics[scale=0.75]{figs/vernier1.png}
    \caption{10 vernier scale divisions are set to coincide with 9 main scale divisions.}
    \label{fig:vernier_1}
\end{figure}

Let us now try to measure the smallest possible distance between 0 and 1 unit. Take a Vernier Calliper and close it completely. If there is no zero-error, the only two readings on the vernier scale and main scale that match are the $0$ and $10$ divisions.

\begin{figure}[!htb]
        \begin{subfigure}[b]{0.5\textwidth}
                \includegraphics[width=0.95\linewidth]{figs/vernier2.png}
                \caption{Just away from 0, when the coinciding division \\is 1, the distance is $1\times\texttt{MSD}-1\times\texttt{VSD}=0.1\, \texttt{units}$.}
                \label{fig:vernier_2}
        \end{subfigure}\hfill
        \begin{subfigure}[b]{0.5\textwidth}
                \includegraphics[width=0.95\linewidth]{figs/vernier3.png}
                \caption{Before crossing 1, the last coinciding division is 9, and the distance is $9\times\texttt{MSD}-9\times\texttt{VSD}=0.9\, \texttt{units}$.}
                \label{fig:vernier_3}
        \end{subfigure}%
        \caption{Measurement with a Vernier Calliper.}
        \label{fig:verniermeasurements}
\end{figure}

\begin{enumerate}
    \item Move the vernier scale \textit{slightly}, until the $1$ on the vernier scale coincides with the nearest main scale division (this is obviously the $1$ on the main scale, see Figure (\ref{fig:vernier_2})).
    
    \item You will notice that the jaws are slightly apart. The distance between them is of course the distance the $0$ on the vernier scale has moved.
    
    \item But this distance is simply the difference between 1 Main Scale Division (\texttt{MSD}) and 1 Vernier Scale Division (\texttt{VSD}), since both the $1s$ coincide!
    
    \item Thus, the spacing between the jaws of the calliper is now $0.1$ units!
    
    \item We have thus been able to measure a distance of $0.1$ units using two scales, one of least count $1$ unit, and the other of least count $0.9$ units!
\end{enumerate}

Similarly, if you went one division further, and had the 2 of the vernier scale coincide with a main scale division, then the distance between the jaws would be $2\times\texttt{MSD}-2\times\texttt{VSD}=0.2\, \texttt{units}$. Thus, if the $n$th vernier scale division coincides with a main scale division, the distance between the jaws is $n \times (\texttt{MSD}-\texttt{VSD})=n \times \texttt{LC}$. Where \texttt{LC} is the Least Count of the Vernier Calliper\footnote{In our case, the Least Count is 0.1 units.}.

Of course, up until right now we have been measuring distances between 0 and 1 unit.  What about an arbitrary distance? Consider the example given in Figure (\ref{fig:vernier_4}). 

\begin{figure}[!htb]
    \centering
    \includegraphics[scale=0.75]{figs/vernier4.png}
    \caption{The main scale reading is more than 2 units. The coinciding vernier scale division is 4 . }
    \label{fig:vernier_4}
\end{figure}

Here is the general procedure to take a reading using a vernier calliper,

\begin{enumerate}
    \item Look where the zero mark on the vernier scale meets the main scale. This gives us our rough reading, usually called the Main Scale Reading or \texttt{MSR}. In the example in Figure (\ref{fig:vernier_4}), the \texttt{MSR} is 2.
    
    \item Now look to find the mark on the vernier scale which most closely meets any mark on the main scale. This is the Vernier Scale Reading or \texttt{VSR}, giving you the most precise digit. In this example, the \texttt{VSR} is 4.
    
    \item The total distance is given by 
    $$\texttt{distance} = \texttt{MSR} + \left(\texttt{VSR}\times\texttt{LC}\right)$$
    
    In our case, $\texttt{distance} = 0.24$ units.
    
    \item Thus, using two scales -- one of least counts $1$ unit and another of least count $0.9$ units -- we have a calliper capable of measuring upto $0.1$ units!
\end{enumerate}

\begin{question}
\paragraph{Question:} Show that the least count of a Vernier Calliper is given by:
\begin{enumerate}
    \item $$\texttt{LC} = \texttt{Least count of main scale} - \texttt{Least count of vernier scale}$$
    \item $$\texttt{LC} = \frac{\texttt{Least count of main scale}}{\texttt{Number of vernier scale divisions}}$$
\end{enumerate}
\end{question}

\subsection{The Micrometer Screw Gauge}

The micrometer screw gauge was invented in the 17th century as an enhancement of the vernier callipers, and is used to measure even smaller dimensions. The screw gauge uses a screw with an accurate amd constant \textbf{pitch} (the amount by which the thimble moves forward or backward for one complete revolution) as an auxiliary scale marked on a rotatable thimble.  

The micrometers in our laboratory have a pitch of 0.01$mm$.  The rotating thimble is subdivided into 100 equal divisions.  The thimble passes through a frame that carries a millimetre scale graduated to 1$mm$.  The jaws can be adjusted by rotating the thimble using the small ratchet knob.  This includes a friction clutch which prevents too much tension from being applied.

\begin{imp}
Only tighten the screw gauge by rotating the ratchet, otherwise you may damage the instrument. Stop rotating after you hear \textbf{three} clicks. \textbf{Do not} tighten the screw any further.
\end{imp}

\begin{question}
\paragraph{Question:} Show that the least count of a screw gauge is given by
$$\texttt{LC} = \frac{\texttt{Pitch of a the screw}}{\texttt{Number of circular scale divisions}}$$
\end{question}

We won't go into using a screw gauge in much more detail as its function very similar to the Vernier Calliper, except with a finer least count.

\section{Theory: Error Analysis}

\subsection{Types of Errors}

``Error'' analysis is a rather glaring misnomer. All measurements have some degree of uncertainty that may come from a variety of sources, from the lack of precision of the measuring instrument to random fluctuations in the environment. The process of evaluating the uncertainty associated with a measurement result is often called uncertainty analysis or error analysis. However, the common definition of an ``error'' being a ``mistake'' is very misleading. However, the terminology is so rampant that we will be forced to use it at times, referring to `error' analysis when we really mean `uncertainty' analysis.


\begin{tip}
Example: if you use a metre scale with a least count (smallest division) of $0.1cm$ upside down to measure a pencil and misread the scale as $94.6cm$ instead of $5.4cm$, this is a \textbf{mistake}.\\

However, if you note down the length as being $5.4cm$, but rightly note that it is not \textbf{exactly} $5.4cm$, but simply that your measuring device does not allow for any more precision, this is a measure of \textbf{uncertainty}. \\

We will expect you to have verified that you haven't done the former, and will take for granted from here on that no mistakes were made in the collection of data.
\end{tip}


The complete statement of a measured value \textbf{must} include an estimate of the level of confidence associated with the value. This allows people to judge the quality of the experiment and allows for comparisons with other similar estimates of the same quantity, or theoretical prediction.  

\begin{imp}
A proper experiment must report both a `best' value and an uncertainty for each measured quantity. You will be expected to do this in all your experiments. 
\end{imp}

Without an uncertainty estimate, it is impossible to answer the basic scientific question: ``Does my result agree with a theoretical prediction or results from other experiments?'' This question is fundamental for deciding if a scientific hypothesis is confirmed or refuted.

\begin{tip}
\paragraph{A cautionary tale:} Neutrinos are weakly interacting nearly massless particles that are theorised to be moving close to -- if not at -- the speed of light.\\

In September 2011, scientists of the OPERA collaboration reported evidence that neutrinos they produced at CERN in Geneva and recorded at the OPERA detector at Gran Sasso, Italy, had travelled \textbf{faster} than light. If this were true, it would have violated one of the pillars of modern physics, the Theory of Relativity.\\

The neutrinos were calculated to have arrived approximately 60.7 nanoseconds sooner than light would have if it had traversed the same distance in a vacuum. Assuming the error were entirely due to random effects, scientists calculated that there was a 0.2-in-a-million chance that this may be a false positive.\\

In March 2012 it was confirmed that a fibre cable was not fully screwed in during data gathering, and this effectively decreased the reported flight time of the neutrinos by 73ns, making them seem faster than light. The corrected difference between the measured and expected arrival time of neutrinos (compared to the speed of light) was approximately 6.5 $\pm$ 15 ns. This is consistent with \textbf{no difference at all}, thus the speed of neutrinos is consistent with the speed of light within the margin of error.
\end{tip}

\subsubsection{Least count errors}

As you have already seen, the least count of a measuring instrument is the smallest value the instrument can resolve. For example, if we use a metre scale to measure an object and find that it is $23.5$cm, the uncertainty here \textbf{due to the measuring instrument} is $0.1$cm. Thus, we can say that the length of the object is given by $l = (23.5 \pm 0.1) cm$ where $\pm \Delta l = \pm 0.1$cm is the uncertainty in measurement due to the least count of the metre scale. This is the value up to which the `best' value of $23.5$ may be specified with any confidence.

In general, such errors are the \textbf{lower bounds} of experimental errors. No error that is less than the least count error need be considered. That being said, if you find other sources of error to be greater than the least count error, then those sources should be used. For example, the stopwatches in the lab have a least count of $0.01$s. However, your reaction time of turning on and off a stopwatch would be at best $0.1$s. Thus, while the least count error is significantly smaller, the \textbf{actual} error in a time measurement using the stopwatch would be of the order of $0.1$s.

\subsubsection{Systematic errors}

Imagine now that the metre scale that you are using to measure the length of the object is made of metal, and as a result expands slightly in summer. If you performed a set of measurements, you would obviously not find the `true' value of the length. However, such an error may be discovered by comparing your measurements from the metal scale with a wooden scale, for example. When using instruments like digital multimeters you obviously rely on it having been calibrated properly. However, there might have been an accident that a student did not report, and as a result there could well be a systematic error for someone unlucky enough to be the one using it next.

Estimating possible errors due to such systematic effects depends on your understanding of your apparatus and the skill you have developed for thinking about possible problems. In this course for the most part such errors will be assumed to be small. However, if you get a value for some quantity that seems rather far off what you expect, you should think about such possible sources more carefully. You could end up trusting a device that you do not know is faulty. This happens very often, to the best of us. 


\subsubsection{Random errors}

Another reason for two measurements not agreeing totally is due to small, uncorrelated variations in the environment or measurement process. For example, while measuring the time period of a pendulum, you might get different readings due to small differences in your reaction time. Similarly, when measuring the height of a cylinder with a vernier calliper, you may see small variations depending on \textbf{where} you measure the height, as it is possible the cylinder was not made to be smooth up to $0.1$mm everywhere.

Such variations, being \textbf{random}, are expected to fall just as often \textit{above} a certain value as \textit{below} it. A truly random source would give you a \textbf{Gaussian} distribution of errors; the mean of this distribution $$\overline{m}_N = \frac{\sum_{i=1}^{N}{m_i}}{N}$$ would approach the `best' value as the number $N$ of well-sampled values increases. The uncertainty of these measurements would then be quantified by the \textbf{standard deviation}\footnote{There is a reason for the $N-1$, it is known as the number of `degrees of freedom' in statistics. Understanding it is not essential for this course.} $$\Delta m_N = \sqrt{\frac{(m_i - \overline{m}_N)^2}{N-1}}$$

more conventionally denoted by $\sigma$.


\subsubsection{Precision, Accuracy, and Statistics}

\begin{enumerate}
    \item \textbf{Accuracy:} is how closely a measurement comes to some `true' value. It determines how well we have eliminated \textbf{\textit{systematic errors}} and mistakes in our measurements.
    \item \textbf{Precision:} is how closely a set of measurements agree \textit{with each other}. It determines how well we have eliminated \textbf{\textit{random errors}} in our measurements.
\end{enumerate}

To use a sports simile, if you're playing football and are accurate, your aim will always take the ball close to or into the goal. If, however, you harbour a personal grudge against the coach and succeed in beaning him on the head every time you get the ball, then you would be \textit{precise}, but not accurate\footnote{It is widely accepted that the point of football get the ball into the goal and not to traumatise your coach, which is why this example works.}.

\textbf{A caveat:} In any Physics experiment worth doing, you usually don't have the answer beforehand. This means you don't have a `true' value with which to compare your answer. This is why it is so important to be both precise (reduce all random errors present in your measurements) and accurate (reduce all systematic errors present in your measurement). You must also, however, come up with ways to check if your answer is accurate \textit{without} trying to resort to a `textbook' solution.

\begin{question}
\paragraph{Question:} In the example of the `superluminal' neutrinos given earlier, was the error of the OPERA team one of inaccuracy, or of imprecision? 
\end{question}

\textbf{A note on statistics:} Consider two sets of measurements of the acceleration due to gravity:

\begin{table}[!htb]
\parbox{.45\linewidth}{
\centering
\begin{tabular}{cc}
\hline
\textbf{S. No.}&$\bm{g}\,\, (ms^{-2})$\\
\hline
1&$9.80\pm0.01$\\
2&$0.70\pm0.01$\\
3&$18.90\pm0.01$\\
4&$15.60\pm0.01$\\
5&$4.10\pm0.01$\\
\hline
Mean&$9.82\pm0.01$\\
\hline
\end{tabular}
\caption{Measurement of $g$ (Set 1)}
}
\hfill
\parbox{.45\linewidth}{
\centering
\begin{tabular}{cc}
\hline
\textbf{S. No.}&$\bm{g}\,\, (ms^{-2})$\\
\hline
1&$9.83\pm0.01$\\
2&$9.80\pm0.01$\\
3&$9.82\pm0.01$\\
4&$9.84\pm0.01$\\
5&$9.83\pm0.01$\\
\hline
Mean&$9.82\pm0.01$\\
\hline
\end{tabular}
\caption{Measurement of $g$ (Set 2)}
}
\end{table}

It would obviously be wrong to go simply by the mean and say that both these sets of data were equally reliable. In fact, stating the mean of the first set as the acceleration due to gravity \textbf{doesn't make sense}. 

Suppose we compared the mean of a data set to its \textit{dispersion} about the mean. If this is small, we could then say that there is some `true' value and that all the different values we measured occurred due to random fluctuations about this `true' value. Since the fluctuations are random, we could assume that they average out to zero, leaving us with a closer estimate to the `true' value than any individual reading.

When the significant source of error is considered to be random, the distribution is a Gaussian, and the measure of dispersion is the \textbf{standard deviation} denoted by $\sigma$. \textit{In such a case}, the error $\delta x$ in the reading could be taken to be $\sigma_x$ without much loss of generality, as it can be shown that there is a 68\% likelihood that an individual measurement will fall within one standard deviation $(\pm\sigma_x)$ of the true value\footnote{Keep in mind, however, the example of superluminal neutrinos we gave earlier: in that case, the error was \textit{assumed} to be random, and their precise data led them to a false positive. The true error turned out to be systematic, leading to \textit{different (and larger) error bars}!}. 

\begin{question}
\paragraph{Question:} In above sets of data (with standard deviations of $7.6\,\,ms^{-2}$ and $0.02\,\,ms^{-2}$ respectively), does the quoted error of $\pm 0.01\,\, ms^{-2}$ make sense
\begin{enumerate}
    \item For Set 1?
    \item For Set 2?
    \item For Both?
    \item For Neither?
\end{enumerate}
Justify your answer quantitatively.
\end{question}

\begin{imp}
The two errors in the lab you will encounter that you cannot remove are \textbf{least count errors} and \textbf{random errors}. Remember to always compare them and \textbf{\textit{take the larger value}}. In Set 1 it makes no sense to say $g$ is specified to $(9.82\pm0.01)\,ms^{-2}$, since all of the values are much farther away than that!
\end{imp}

\begin{question}
\begin{enumerate}
    \item \textbf{Question:} Multiple measurements with the same instrument increases the
    \begin{enumerate}
        \item Accuracy
        \item Precision
        \item Both
    \end{enumerate}
    
    \item \textbf{Question:} Consider the following data-table~\\
    \begin{tabular}{cccccc}
    \hline
    \textbf{S. No.}&1&2&3&4&5\\
    \hline
    \textbf{Time Period} ($s$)& $1.2\pm0.1$&$1.2\pm0.1$&$1.2\pm0.1$&$1.2\pm0.1$&$1.2\pm0.1$\\
    \hline
    \end{tabular}~\\~\\
    The standard deviation is $0.0$. Would it be right to say that $T_\text{avg} = 1.2\pm 0.0$?
\end{enumerate}
\end{question}

However, keep in mind there are many cases where the mean  \textbf{does not} represent some true value\footnote{For example, even if the average number of siblings every student has is $1.574$, there is no student who has a non-integer number of siblings.}. There are also cases where the standard deviation contains physical information. Such cases are usually a result of statistical phenomena.

\begin{tip}
\begin{enumerate}
    \item In a coin-toss experiment with a large number of tosses, $\sigma$ gives you a measure of the bias of the coin.
    \item In a random walk, $\sigma$ can be a measure of the diffusion coefficient $D$.
    \item In shot-noise --  the statistical fluctuations of current due to the actual number of electrons flowing in the conductor per unit time -- $\sigma$ gives you a measure of Boltzmann's constant $k_B$.
\end{enumerate}
\end{tip}


\subsection{Reporting Errors}
\subsubsection{Significant figures}

The significant figures of a number are the digits in its representation that contribute to the precision of the number. In practice, we assume that all digits used to write a number are significant (except leading zeroes\footnote{Non-leading zeros are considered to be significant. If you write a number as 1,200, we assume there are four significant digits. If you only mean to have two or three, then it is best to use scientific notation: $1.2 \times 10^3$ or $1.20 \times 10^3$ . Leading zeros are not considered significant: 0.55 and 0.023 have just two significant figures.}).

\textbf{Results of simple calculations should not increase the number of significant digits}. Calculations transform our knowledge; they do not increase it! The rounding should be performed at the final step of a calculation to prevent rounding errors at intermediate steps from propagating through your work but \textbf{only} one or two
extra digits suffice to prevent this.

If you measure a value on a two-digit digital meter to be 1.0 and another value to be 3.0, it is incorrect to say that the ratio of these measurements is 0.3333333. The two values are not exact numbers with infinite precision. Since they each have two significant digits, the correct number to write down is 0.33\footnote{If this is an intermediate result, then 0.333 or 0.3333 are preferred, but the final result must have two significant digits.}.

\begin{imp}
\textbf{Do not write significant figures beyond the first digit of the error on the quantity}. Giving more precision than this to a value is not only irrelevant, \textit{it is misleading}.\\

If you're told you're using FAR too many digits, please do not try to use the excuse, ``That's what the computer gave me.'' \textbf{You} are in charge of presenting your results, not the computer!
\end{imp}


\subsubsection{Error propagation}

Experiments worth carrying out rarely measure only one quantity. Typically we measure two or more quantities and then `fold' them together in some equation or equations to determine some other quantity that we believe to depend on them. It is thus imperative that we understand how uncertainties in certain measured quantities `propagate' into other derived quantities.

For our analysis, let us assume that

\begin{itemize}
    \item $x$ and $y$ are measured quantities with uncertainties $\delta x$ and $\delta y$ respectively. These errors are considered to be \textbf{uncorrelated}, meaning that $\delta x$ and $\delta y$ are independent\footnote{For example, the precision of measuring the length of your simple pendulum has no effect on the precision of measuring time.}.
    
    \item $c$ is a constant known to known absolutely precisely (or with negligible uncertainty).
    
    \item $z$ is a quantity \textit{derived} from $x$ and $y$ and possessing a `propagated' uncertainty $\delta z$.
\end{itemize}

The formulae used to compute errors are usually not completely understood: a variety of different formulae are used in different cases, and the reasons \textbf{why} are usually lost on students. It turns out that there is only \textbf{one} way to add (uncorrelated) errors, which one can manipulate to get the rest:

\begin{imp}
If $z = x + y$, their (uncorrelated) errors add \textbf{in quadrature}:
\begin{equation}
    \delta z = \sqrt{\left(\delta x\right)^2+\left(\delta y\right)^2}
    \label{quadrature}
\end{equation}
\end{imp}

The reason for errors adding in quadrature is one that comes from statistics; it is essential that the errors be \textbf{independent} of each other (uncorrelated)\footnote{It also makes sense:  errors could be positive and some negative, simply adding them could conceivably give a smaller number (or even zero!). Thus, the next best thing is to add their \textit{squares}. }

Here is another motivation\footnote{This is \textbf{not} an explanation of why this is true!}: you could imagine two uncorrelated measurements to represent two axes ($x$ and $y$) that are orthogonal to each other (see Figure (\ref{fig:quadrature}). Imagine that you want to specify a point $z = (x,y)$: given that there is an uncertainty in both the coordinates $x$ and $y$, the point $z$ is uncertain \textbf{at most} by $\delta z = \sqrt{\left(\delta x\right)^2+\left(\delta y\right)^2}$.

\begin{figure}
    \centering
    \includegraphics[scale=0.5]{figs/quadrature.png}
    \caption{Two points $(x,0)$ and $(0,y)$, with uncertainties $\delta x$ and $\delta y$ respectively, can be summed to get a point $z=(x,y)$ with uncertainty $\delta z$. }
    \label{fig:quadrature}
\end{figure}

This formula can then be used to get the uncertainties of more complicated relations. For example, consider $z = xy$. In this case, the two quantities are \textbf{multiplied}, and so we can't use the above formula as is. We could, however, take the log, and get $\log{z} = \log{x} + \log{y}$. This is of the form $u = v + w$. We could then apply Equation (\ref{quadrature}), and find that $$\delta \left(\log{z}\right) = \sqrt{\left(\delta \left(\log{x}\right)\right)^2 + \left(\delta \left(\log{y}\right)\right)^2}$$

The uncertainty in $\log{z}$ can easily be related by taking the derivative\footnote{A differential is by definition the variation of function when its parameter changes by a small amount. We want to find how much $\log{z}$ changes when $z$ changes by $\delta z$. We use $\delta$ instead of $\dd$, since the variation is not truly \textit{infinitesimal}.}.

\begin{equation*}
    \dd({\log{z}}) = \frac{\dd z}{z} \quad \implies \quad \delta (\log{z}) = \frac{\delta z}{z}
\end{equation*}

Thus, 

\begin{equation}
    \frac{\delta z}{z} = \sqrt{\left(\frac{\delta x}{x}\right)^2 + \left(\frac{\delta y}{y}\right)^2}
\end{equation}

\begin{tip}
The following rules should exhaust most of the common cases that you will be exposed to in your undergraduate labs. Everything here can be generalised simply to a a \textit{set} of measurements $x_i$ with uncertainties $\delta x_i$.

\begin{enumerate}
    \item \textbf{Addition or subtraction by a constant:} If $z = c \pm x$, then 
    \begin{equation}
        \delta z = \delta x
    \end{equation}
    
    
    \item \textbf{Multiplication by a constant:} If $z = c x$, then 
    \begin{equation}
        \delta z = c\delta x
    \end{equation}
    
    \item \textbf{Addition or subtraction of two measured quantities:} If $z = x \pm y$, then 
    
    \begin{equation}
        \delta z = \sqrt{(\delta x)^2 +(\delta y)^2}
    \end{equation}
    
    \item \textbf{Multiplication or division of two measured quantities:} If $z = xy$ or $z = \frac{x}{y}$, then 
    
    \begin{equation}
        \frac{\delta z}{z} = \sqrt{\left(\frac{\delta x}{x}\right)^2 + \left(\frac{\delta y}{y} \right)^2}
        \label{relerror}
    \end{equation}
    
    \item \textbf{A measured quantity raised to a power:} If $z = x^c$, then
    
    \begin{equation}
        \frac{\delta z}{z} = c \frac{\delta x}{x}
        \label{powerror}
    \end{equation}
    
\end{enumerate}
\end{tip}

\begin{question}
\paragraph{Question:} Prove Equation (\ref{powerror}).~\\

\paragraph{Question:} If $z = x^2 = x \times x$, we get different answers if we use the Power Rule (Equation (\ref{powerror})) or the Product Rule (Equation (\ref{relerror})). Which of the two is correct? Why? ~\\

\paragraph{Question:} Calculate the uncertainty in $z$ if
\begin{enumerate}
    \item $z = \frac{1}{x}$
    \item $z = \frac{x}{1+x}$
    \item $z = \frac{x}{x+y}$
\end{enumerate}
\paragraph{Hint:} The last one is slightly hard. You will first write it as $z = \frac{1}{1 + u}$, where $u = y/x$. Then, calculate $\delta z$ in terms of $\delta u$, and only then $\delta u$ in terms of $\delta x$ and $\delta y$.
\end{question}

\section{Apparatus}

\begin{enumerate}
    \item A pair of Vernier Callipers
    \item A Screw Gauge
    \item A set of ball bearings
    \item A set of aluminium cylinders
    \item A conical measuring flask
    \item A container with a spout
\end{enumerate}

\section{Description}

\begin{enumerate}
    \item In \textbf{Part A} you will measure the radius, height, and volume of different cylinders. You will then plot an appropriate graph from which you will extract the value of $\pi$.
    
    \item In \textbf{Part B} you will measure the radius and volume of different ball bearings. You will then plot an appropriate graph from which you will extract the value of $\pi$.
    
\end{enumerate}

\section{Suggested Procedure}

\subsection{Part A}

\begin{enumerate}
    \item Decide on the best instrument and method to measure the following properties of the cylinders:
    \begin{enumerate}
        \item Radius
        \item Height
        \item Volume
    \end{enumerate}
    
    \begin{question}
        \paragraph{Question:} How and where would you measure the radius of the cylinders? How many trials would you take?
    \end{question}
    
    \item Be sure to take a sufficient number of \textbf{well-sampled} readings.
    
    \item Decide on an appropriate graph between the measured quantities from which you can extract the numerical value of $\pi$.
\end{enumerate}

\subsection{Part B}

\begin{enumerate}
    \item Decide on the best instrument and method to measure the radii and the volumes of the bearings.
    
    \item Be sure to take a sufficient number of \textbf{well-sampled} readings.
    
    \item Decide on an appropriate graph between the measured quantities to extract the numerical value of $\pi$.
\end{enumerate}

\begin{question}
\paragraph{Question:} How does your estimate compare with the known value of $\pi \approx 3.14159$?
\end{question}



\newpage

\chapter{Plotting}
\section{Objectives}

\begin{enumerate}
    \item To learn to plot graphs using Python and Microsoft Excel.
\end{enumerate}

\section{Introduction}


\section{Theory: Curve Fitting with Python}

In this section you will learn how to:

\begin{enumerate}
    \item Import data from CSV files or spreadsheets
    \item Basic plot, axes, units, points, etc.
    \item Error Bars
    \item Fitting a curve
\end{enumerate}


\section{Apparatus}



\section{Description}



\section{Suggested Procedure}



\newpage

\vbox{
\textcolor{Blue}{\part{Experiments}}
\tikz[remember picture,overlay]\node[shift={(-1,1)},opacity=0.6] at (current page.south east) {\includegraphics[width=17.5cm]{logo}};
}

\renewcommand{\chaptername}{Experiment}

% \title{Diffraction of Light}
% \author{An Introduction to Physics through Experiments}
% \date{}
% \maketitle

\chapter{Diffraction of Light}

\section*{Objectives}

\begin{enumerate}
    \item To determine the wavelength $\lambda$ of the light emitted by a laser source by studying the diffraction of light due to plane diffraction gratings.
    \item To determine the width of the given single slit by studying its diffraction pattern.
    \item To determine the diameter of a given wire by studying its diffraction pattern.
    \item To determine the size of the circular aperture by studying its diffraction pattern.
\end{enumerate}



\section*{Apparatus}

\begin{enumerate}
    \item A 10 mW semiconductor red laser source.
    \item A 5 mW DPSS green laser source.
    \item A set of necessary mounts.
    \item A set of plane diffraction gratings of different grating spacings.
    \item A single helix (spring) set in a holder
    \item A double helix set in a holder
    \item Measuring tapes.
    \item A holder for grating.
    \item A set of screens.  
    \item A single slit of fixed width mounted on a slide.
    \item Two multiple slits mounted on slides.
    \item Circular apertures mounted on slides.
    \item A spirit level
\end{enumerate}

\section*{Introduction}
	 
In \textit{Opticks} (1704), Issac Newton wrote, ``Light is never known to follow crooked passages nor to bend into the shadow''. He explained this by describing how particles of light always travel in a straight line, and how objects kept in the path of the light cast a shadow because the particles can never spread out behind the object. However, a set of experiments on the propagation of light through small apertures performed by Francesco Grimaldi, Augustine Fresnel, Thomas Young and a few others firmly established that light actually enters into the shadow region with a definite pattern when it passes through around an edge. The resulting pattern depends on the relative size of the aperture or obstacle and the wavelength of light. If the size is much larger than the wavelength, the bending will be almost unnoticeable. However, if the two are similar in size, the diffraction will be considerable.    
     
In this experimental problem, we will use a low power solid-state laser as a source of an intense beam of monochromatic light. When light from a distant source (or a laser source) passes around a thin aperture or through a narrow aperture and is then intercepted by a viewing screen, the light produces a pattern on the screen called a \textit{diffraction pattern}. When such a beam is incident on various diffracting components like a plane diffraction grating, a single slit, a wire mesh or a two-dimensional diffraction grating, the light emerging from these components show a variety of interesting diffraction patterns. This pattern consists regions of maximum and minimum intensities, which characterise the diffracting object. 

\section*{Theory}


\begin{imp}

\begin{center}
    \textbf{Babinet's Principle}
\end{center}

Two objects are called \textit{complementary} if one of them is transparent where the other is opaque and opaque where the other is transparent. An example of this is given in Figure (\ref{fig:complementary}).~\\

\textbf{Babinet's principle} states that

\begin{center}
``Complementary objects \textbf{produce the same diffraction pattern}, except for the intensity of the central maxima.''
\end{center}

This is a highly non-intuitive result, and one that you may study in some of your later courses. The mathematics of this are far beyond the scope of this lab session. 
Henceforth we will not differentiate between the patterns made by such complementary objects. We will thus speak of a single slit (or of a thin wire) having the `same' diffraction pattern, keeping in mind that the intensities of the central maxima are different.
\end{imp}

\begin{figure}[!htb]
    \centering
    \includegraphics[scale=0.6]{figs/complementary.png}
    \caption{Transparency of a thin wire and of a single slit.}
    \label{fig:complementary}
\end{figure}

\begin{question}
\paragraph{Question:} What are the complementary objects of the following:

\begin{enumerate}
\item A transparent sheet of paper.
\item A circular aperture of diameter $d$.
\end{enumerate}
\end{question}



\subsection*{The Single Slit (or Thin Wire)}

When a (monochromatic) beam of light such as a laser is incident on a narrow single slit, the light emerging from the slit shows a diffraction pattern on a screen. The distribution of the intensity of light received on a screen show a pattern of varying intensity consisting of a bright central maximum with alternate minima and maxima of decreasing intensity on either side, known as a \textit{Fraunhofer diffraction pattern}.

Similarly, instead of a slit of width $a$, supposing we had a \textit{wire} of diameter $a$ placed as an obstruction to the laser beam. The resulting intensity pattern as observed on a screen almost exactly the same as that observed for a single slit. This intensity distribution can be written as a function of the angle $\theta$ as

\begin{equation*}
    I(\theta) = I(0) \left( \frac{\sin \beta}{\beta} \right)^2, \quad \quad  \beta = \frac{\pi a \sin \theta}{\lambda}
\end{equation*}

This function has been represented in Figure (\ref{fig:singleslit}).

\begin{figure}[!htb]
    \centering
    \includegraphics[width=0.75\textwidth]{figs/singleslit.png}
    \caption{The diffraction pattern produced by a single-slit aperture (or thin wire).}
    \label{fig:singleslit}
\end{figure}


Moving away from the central spot, when $\sin\beta=0$ (but $\beta \neq 0$), the intensity vanishes! The positions of these \textbf{minima} (zeros) of the intensity distribution pattern are given by the relation 

\begin{equation}
   a \sin{\theta_m} = \pm m \lambda  \quad\quad\quad \text{for    m  = 1, 2, 3,}\hdots,
   \label{singleslit}
\end{equation}

where $\lambda$ is the wavelength of the incident light, $a$ is the width of the slit and $\theta_m$ is the angle corresponding to $m$th minimum. Here the $\pm$ refers to either side of the central spot ($\theta = 0$).


\subsection*{The Double Slit}

We can now imagine two single slits of the same width next to each other. Such an arrangement is called a double-slit aperture. We could think of this aperture (shown in Figure (\ref{fig:doubleslit})) as a combination of a single slit of width $d$, and a thin wire of width $a$. 

This situation is slightly more complicated than the previous one; the thin obstacle of width $a$ will produce a diffraction pattern like the one in the previous section. However, the light from the two slits on either side will \textbf{interfere}, producing an overall interference pattern which also has its own maxima and minima! The mathematics of this is too difficult to understand right now; you will cover it in a later laboratory.

The resultant intensity distribution is given by   

\begin{equation}
    I(\theta) = I(0) \cos^2\delta \left( \frac{\sin \beta}{\beta} \right)^2, \quad \quad  \beta = \frac{\pi a \sin \theta}{\lambda}, \quad \delta = \frac{\pi d \sin \theta}{\lambda}
\end{equation}

For a screen placed at a large distance $D$ from the wire, the positions of the minima on the screen are observed at 

\begin{equation}
    \begin{aligned}
        x_{\pm n} = \pm n \frac{\lambda D}{a}, &\quad& \text{due to diffraction},\\
        x_{\pm m} = \pm \left( m - \frac{1}{2}\right) \frac{\lambda D}{d}, &\quad& \text{due to interference}
    \end{aligned}
    \label{doubleslit}
\end{equation}


\begin{figure}[!htb]
    \centering
    \includegraphics[width=0.75\textwidth]{figs/doubleslit.png}
    \caption{The diffraction pattern produced by a double-slit aperture is a combination of two patterns (diffraction due to a single wire and interference due to two slits).}
    \label{fig:doubleslit}
\end{figure}



\begin{question}
\paragraph{Question:} Look at the above condition for the fringes produced due to diffraction in Equation (\ref{doubleslit}). Is it different from the result obtained in the previous section (Equation (\ref{singleslit}))? If yes, why? If no, why not?~\\
\paragraph{Question:} What is the ``complementary'' object to a double-slit?
\end{question}



\subsection*{Plane Diffraction Grating}

A transmission diffraction grating consists of a large number of slits separated from one another by an opaque region. The grating concentrates the diffracted light along a particular direction in contrast to the single slit, which has a rather broad diffraction maximum. The maxima (bright intense spots) produced by a grating are usually called the principal maxima. They are quite intense and are also widely separated; what cannot be detected visually are the large number of secondary maxima which lie between neighbouring principal maxima.

The expression, relating wavelength $\lambda$ of light used and the grating spacing $d$, with angle of deviation $\theta$ is
\begin{equation*}
    d \sin{\theta_m} = m \lambda,  \quad\quad\quad \text{for    m  = 1, 2, 3,}\hdots
\end{equation*}

In the above expression, $m$ represents the order of \textbf{maxima} points and the angle $\theta_m$  corresponds to  $m$th  order maximum intensity point. This relation is valid for a single slit and for the wire-like obstacle.

\begin{figure}[!htb]
    \centering
    \includegraphics[width=0.75\textwidth]{figs/grating.png}
    \caption{The diffraction pattern produced by a plane diffraction grating.}
    \label{fig:grating}
\end{figure}


\subsection*{The Circular Aperture}

The diffraction pattern due to a circular aperture (known as an \textit{Airy diffraction pattern}) is similar to a single slit diffraction but the mathematics involved is more complicated which gives the expression nearly identical to that of the single slit. Hence we may apply the same expression to the diffraction due to a circular aperture, 

\begin{equation*}
    d \sin{\theta_m} = \overline{m} \lambda,  \quad\quad\quad \text{for    m  = 1, 2, 3,}\hdots
\end{equation*}

where $d$ is the diameter of the circular aperture and $\theta_m$ is the angle of deviation for the $m$th dark ring. The variable $\overline{m}$ has the following values:

\begin{equation*}
    \begin{aligned}
        m = 1 &\quad& \overline{m}=1.22\\
        m = 2 &\quad& \overline{m}=2.23\\
        m = 3 &\quad& \overline{m}=3.23\\
        m = 4 &\quad& \overline{m}=4.24\\
    \end{aligned}
\end{equation*}

\subsection*{Single and double helices:}
  
\begin{figure}[!htb]
    \centering
    \begin{subfigure}[b]{0.45\textwidth}
                \centering
                \includegraphics[width=0.75\textwidth]{figs/singlehelix.png}
                \caption{Schematic of a single helix (like RNA)}
                \label{fig:singlehelix1}
        \end{subfigure}%
        \begin{subfigure}[b]{0.45\textwidth}
                \centering
                \includegraphics[width=0.75\textwidth]{figs/doublehelix.png}
                \caption{Schematic of a double helix (like DNA)}
                \label{fig:doublehelix1}
        \end{subfigure}
\end{figure}

Now consider a set of four identical wires, the net intensity distribution is a combination of diffraction from each wire and interference due to pairs of wires and hence depends on $a$, $d$ and $s$ (see Figure \ref{fig:fourwires}).

In other words, the combination of three different intensity patterns is observed.

\begin{figure}[!htb]
    \centering
    \begin{subfigure}[b]{0.45\textwidth}
                \centering
                \includegraphics[width=0.75\textwidth]{figs/fourwires.png}
                \caption{Projection of four identical wires}
                \label{fig:fourwires}
        \end{subfigure}%
        \begin{subfigure}[b]{0.45\textwidth}
                \centering
                \includegraphics[width=0.75\textwidth]{figs/doublehelix-schema.png}
                \caption{The double helix given in the sample.}
                \label{fig:doublehelix2}
        \end{subfigure}
        \caption{Diffraction patterns of obstacles with three length scales.}
\end{figure}

\section*{Description}

In \textbf{Part A}, you will determine the wavelength of light of a laser using a set of diffraction gratings.

In \textbf{Part B}, you will use the laser whose wavelength you have just determined to measure the width of a single slit.

In \textbf{Part C}, you will study diffraction pattern of a circular aperture.

In \textbf{Part D (\textit{optional})}, you will study the diffraction pattern of a single helix.

In \textbf{Part E (\textit{optional})}, you will study the diffraction pattern of a double helix.


\section*{Procedural Instructions}

\subsection*{Part A}

Observe effect of colour and also of white light on the diffraction pattern obtained by a suitable grating. Then choose an appropriate diffraction grating and perform the measurements to determine the wavelength $\lambda$ of the laser. 

\begin{question}
\paragraph{Question:} Estimate the error in the value of the wavelength of light.~\\

\paragraph{Question:} What are the sources of error in the above-determined value of $\lambda$? What measures should be taken to minimise these errors? ~\\

\paragraph{Question:} Tilt the grating at an angle. How does this affect the diffraction pattern?
\end{question}

Now repeat this procedure for different gratings (with different values of $d$), and calculate wavelength $\lambda$ as accurately as possible.

\subsection*{Part B}

Design and perform the necessary experiment with a single slit of fixed width and determine the width d of the given single slit.

\begin{question}
\paragraph{Question:} Tilt the slit at an angle. How does this affect the diffraction pattern?
\end{question}


\subsection*{Part C}

Now take the given circular aperture as the diffracting object and determine the diameter of the circular aperture.

\subsection*{Part D}

Study of the diffraction pattern due to a helical spring and determine pitch of the spring and thickness of its wire. 


\begin{question}
\paragraph{Question:} Can you explain the form of the diffraction pattern observed? Does this part of the experiment relate in any way to Part B? 
\end{question}

\subsection*{Part E}

Study of the diffraction pattern due to a double helix (as in our DNA) and determine all its parameters, as shown in Figure (\ref{fig:doublehelix2}).

\begin{question}
\paragraph{Question} Describe the diffraction pattern obtained if you use a laser source to illuminate
\begin{enumerate}[label=(\alph*)]
    \itemsep0em
    \item A fine wire mesh,
    \item A square aperture,
    \item A rectangular aperture.
\end{enumerate}
\end{question}



\subsection*{Precautions}

\begin{enumerate}
    \item Never look directly at a laser beam with the naked eye. It may damage the eye permanently.
    \item Never point the laser at anyone else, for the same reason.
    \item Never point an optical device at a laser beam. It could damage the internal sensors.
    \item Never place highly reflective objects (such as rings, watches, and glassware) in the path of the laser beam.
    \item For proper working of laser, it should be kept on throughout. Do not put it off until you complete all your readings, but if you do not need the laser beam for measurements or alignment, use a light-blocking screen to block the  beam.
    \item Treat the laser source as you would any other electrical device: It should never be tampered with while the power cord is connected.
\end{enumerate}


\section*{References}

\begin{enumerate}
\itemsep0em
    \item Eric Stanley, Am. J. Phys., Vol.- 54, No.-10, October 1986, pp. 952.
    \item F.A. Jenkins, H.E. White, \textit{Fundamentals of Optics}, Third Edition, Mc-Graw Hill Kogakusha Ltd., Toyko, Japan, 1957, pp. 288-309, 328-350.
    \item F.W. Sears, \textit{Optics}, Third Edition, Asia Publishing House, 1958, pp 221-252.
    \item R.W. Ditchburn, \textit{Light}, Second Edition, The English Language Book Society and Blackie \& Son Ltd., 1963, pp 162-237.
    \item John Beynon, \textit{Introductory University Optics}, Prentice-Hall of India Pvt. Ltd., New Delhi (India), 1998, pp 158-190.
    \item Rajpal S. Sirohi, \textit{Wave Optics and its Applications}, Orient Longman Limited, (India), 1993, pp 169-210.
\end{enumerate}



\newpage
% \title{The Soft Massive Spring}
% \author{An Introduction to Physics through Experiments}
% \date{}
% \maketitle

\chapter{The Soft Massive Spring}

\section*{Objectives}

\begin{enumerate}
\item To determine the spring constant and the mass correction factor for the given soft massive
spring by static (equilibrium extension) method.
\item To determine the spring constant and the mass correction factor for the given soft massive
spring by dynamic (spring mass oscillations) method.
\item To determine the frequency of oscillations of the spring with one end fixed and the other
end free i.e. zero mass attached.
\item To study the longitudinal stationary waves and to determine the fundamental frequency of
oscillations of the spring with both the ends fixed.
\end{enumerate}

\section*{Apparatus}

\begin{enumerate}[label=\arabic*)]
\itemsep0em
\item A set of soft massive springs
\item A long and heavy retort stand with a clamp at the top end 
\item A set of masses with hooks
\item A signal generator (\textit{Equip-tronics QT-210})
\item A dual output power amplifier with the connecting cords
\item A mechanical vibrator
\item A digital multimeter (\textit{Victor VC97})
\item A digital stopwatch (\textit{Racer})
\item Measuring tapes
\item A set of measuring scales (1.0 m, 0.6 m and 0.3 m)
\end{enumerate}

\section*{Introduction}

A spring is a flexible elastic device which stores potential energy by the straining of the bonds between the atoms of the elastic material of which it is made. A variety of springs are available which are designed and fabricated to suit the various mechanical systems. Some common types of springs are compression springs, extension springs and torsion springs. 

Robert Hooke, a 17th century physicist, studied the behaviour of springs under different loads. He established an equation, now known as Hooke’s law of elasticity which states that the amount by which a material body is deformed (the \textit{strain}) is linearly proportional to the force causing the deformation (the \textit{stress}). When applied to a spring, Hooke’s law implies that the restoring force is linearly proportional to the equilibrium extension. In other words,  $$F = - k x,$$ where $F$ is the restoring force exerted by the spring, $x$ is the displacement from the equilibrium position and $k$ is called the spring constant.\footnote{The negative sign indicates that the force $F$ is opposite in direction to the extension $x$, hence the term `restoring'.} For this equation to be valid, $x$ needs to be below the elastic limit of the spring. If $x$ is more than the elastic limit, the spring will exhibit `plastic behaviour', where the atomic bonds in the material of the spring get broken or rearranged and the spring no longer returns to its original state. It may be noted that the potential energy $U$ stored in a spring is given by $$U= \frac{1}{2} k x^2$$

Depending on the value of the spring constant, a spring can be called \textit{soft} or \textit{hard}. A spring may also be considered to be massless or massive, depending on the mass that needs to be attached to it to get a considerable extension in the spring. Sometimes springs are also categorised by the ratio of spring constant to the mass of the spring $(\nicefrac{k}{m_s})$. A soft massive spring has a low spring constant and its mass cannot be neglected. 

Ideal springs are considered to be massless. Hung vertically without any mass attached, an ideal spring shows no extension. Similarly, with an attached mass $M$ the spring is found to oscillate with a characteristic time period $$T = 2\pi \sqrt{\frac{M}{k}}$$ Practically (for example in the determination of its spring constant), we usually neglect the mass of the spring. However, if the spring extends under its own weight, its mass cannot be neglected and therefore the extension requires a correction. Similarly, they oscillate without any attached mass, implying that the standard formula for the time period of oscillations of a spring needs modification. The corrected formulae have been worked out and, interestingly, one finds that the correction factors $m_\text{corr}$ due to the spring's mass $m_s$ in these two cases are not the same. In this problem, we will experimentally study and verify the modified formulae. 

An extended soft massive spring clamped at both the ends can be assumed to be a uniformly distributed mass system. It has its own natural frequencies of oscillations (corresponding to different normal modes) like a hollow pipe closed at both the ends. Using the method of resonance we will excite and study different normal modes of vibrations of the spring. Here the longitudinal stationary waves will be set up on the extended soft massive spring. 

\section*{Description}

In \textbf{Part A}, we will use the static method, where the equilibrium extension of a given spring will be measured for different attached masses and the spring constant and the mass correction factor will be determined. 

In \textbf{Part B}, we will use the dynamic method, where different masses will be attached to the lower end of the spring with its upper end fixed and corresponding time period of oscillations for such a spring-mass system will be measured. The frequency of oscillations of the spring with the upper end fixed and the lower end free (i.e. with no attached mass) will be determined graphically. 

In \textbf{Part C}, we will use a mechanical vibrator to force oscillations on the spring and excite different normal modes of vibrations of the spring. Thus the longitudinal stationary waves will be set up on the spring. We will measure the frequencies of the excitations corresponding to different normal modes. From these, the fundamental frequency of oscillation with both the ends fixed will be determined. We will compare this frequency with the frequency of oscillations with one end fixed and the other end free as determined earlier in Part B.



\section*{Theory}

\subsection*{Part A}

Let $L_0$ be the length of the spring when the spring is kept horizontal under no tension, $m$ be the mass attached to the free end of the spring, $L_m$ be the length of the spring when the mass is attached at its lower end, $S_m$ be the equilibrium extension of the spring for mass $m$, $m_s$ be the mass of the spring, $k$ be the spring constant and $g$ be the acceleration due to gravity.

Thus,

\begin{equation}
S_m = L_m - L_0
\end{equation}

We can determine the expression for $S_m$, by taking extension of a small element and integrating over the total length of the spring,

\begin{equation}
S_m = \left( m + \frac{m_s}{2} \right)\left(\frac{g}{k} \right)
\end{equation}

The factor $(\nicefrac{m_s}{2})$ is the mass correction factor in the \textbf{static} case.


\subsection*{Part B}
In case of the soft massive springs, we cannot neglect the mass of the spring since these springs can oscillate without any attached mass. We thus need to modify the earlier expression for $T$. This can be done using the principle of conservation of energy, $$\text{Potential Energy + Kinetic Energy = constant.}$$

The modified expression is found to be

\begin{equation}
T = 2\pi \sqrt{\cfrac{\left(m + \cfrac{m_s}{3}\right)}{k}}
\end{equation}

The factor $(\nicefrac{m_s}{3})$ is the mass correction factor in the \textbf{dynamic} case. Note that this factor is different from the mass correction factor in the previous (static) case.

The corresponding frequency is given by $$f_0 = \frac{1}{T}.$$


\subsection*{Part C}

As mentioned earlier, the extended spring has its own natural frequencies like a hollow pipe closed at both the ends. Note that both the ends of the spring may be taken to be fixed in the case in question: the upper end is fixed in any case and the amplitude of the lower end is small, as compared to the extended length of the spring, and can be taken to be nearly zero. The natural frequencies correspond to stationary waves; their wavelengths are given by 

\begin{equation*}
\lambda_n = \left( \frac{2}{n}\right) L , \quad \quad n=1,2,3\hdots
\end{equation*}

We know that the speed $v$ of the waves on the spring follow the equation $$v = f_n \lambda_n$$ Thus,

\begin{equation}
\begin{aligned}
f_n = \frac{v}{\lambda_n} &= \frac{v n}{2L}\\
f_1 = \frac{v}{2L}, \quad f_2 = \frac{v}{L},\quad &\hdots,\quad f_n = n f_1
\end{aligned}
\end{equation}

\section*{Experimental Setup}

For Parts A and B, you will need a soft massive spring, a retort stand with a clamp, a set of
masses, a measuring tape or scales and a digital stopwatch. For Part C, you will need a soft massive spring, a long and heavy retort stand with a clamp at the top and a mechanical vibrator clamped near the base of the stand. We will also need a function generator. In this case, the soft massive spring should be clamped at the upper end of the stand. The lower end of the spring should be clamped to the crocodile clip fixed at the centre of the mechanical vibrator. The lower end of the spring will be subjected to an up and down harmonic motion using the mechanical vibrator. It must be ensured that the amplitude of this motion is small enough so that these ends could be considered to be fixed.

\subsection*{Warnings}

\begin{itemize}
\item Do not extend the spring beyond the elastic limit. Choose the value of the maximum mass that may be attached to the lower end of the given spring carefully.
\item Keep the amplitude of oscillations of the spring-mass system just sufficient to get the required number of oscillations.
\item The amplitude of vibrations should be carefully adjusted to the required level using the amplitude knob of the function generator so as to not blow the fuse in the power amplifier. The brighter and more frequently the indicator LEDs flash indicates that the fuse is close to blowing.
\end{itemize}

\section*{Procedural Instructions}

\subsection*{Part A}
\begin{enumerate}
\item Measure the length $L_0$ of the spring keeping it horizontal on a table in an unstretched (all the coils touching each other) position.
\item Hang the spring to the clamp fixed to the top end of the retort stand. The spring extends under its own weight.
\item Take appropriate masses and attach them to the lower end of the spring. 
\item Measure the length $L_m$ of the spring in each case. (For better results you may repeat
each measurement two or three times.) Thus determine the equilibrium extension $S_m$ for each value of mass attached.
\item Plot an appropriate graph and determine the spring constant $k$ of the spring and the mass of the spring $m_s$.
\end{enumerate}

\begin{question}
\paragraph{Question:} State and justify the selection of variables plotted on the $x$ and $y$ axes. Explain the observed behaviour and interpret the $x$ and $y$ intercepts.
\end{question}


\subsection*{Part B}

\begin{enumerate}
\item Keep the spring clamped to the retort stand.

\item Try to set the spring into oscillations without any mass attached, you will observe that
the spring oscillates under the influence of its own weight.

\item Attach different masses to the lower end of the spring and measure the time period of oscillations of the spring mass system for each value of the mass attached. You may measure the time for a number of oscillations to determine the average time period.

\item Perform the necessary data analysis and determine spring constant $k$ and the mass of the spring $m_s$ using the above data. Compare these values to those obtained in the Part A.

\item Also determine frequency $f_0$ for zero mass attached to the spring from the graph.

\end{enumerate}


\subsection*{Part C}

\begin{enumerate}
\item Keep the spring clamped to the long retort stand.

\item Clamp the lower end of the spring to the crocodile clip attached to the vibrator.

\item Connect the output of the function generator to the input of the mechanical vibrator
through the power amplifier, using the BNC cable.

\item Starting from zero, slowly go on increasing the frequency of vibrations produced by the vibrator by increasing the frequency of the sinusoidal signal generated by the function generator. At one particular frequency you will observe that the midpoint of the spring oscillate with large amplitude indicating an antinode. (You may use a small piece of tissue paper to observe the amplitude at the antinode.) This is the fundamental mode (first harmonic) of oscillation of the spring. Adjust the frequency to get the maximum possible amplitude at the antinode. Measure and record this frequency using the frequency setting on the multimeter.

\item You will find, however, that it is easier to locate nodes (places where the spring does not move) than antinodes. Increase the frequency further and observe higher harmonics identifying them on the basis of the number of loops you can see between the fixed ends. (If you see $n$ nodes -- or fixed points -- between the endpoints, there are $n+1$ loops.)

\item Plot a graph of frequency for different number of loops versus the number of loops. Determine this fundamental frequency $f_1$ from the slope of this graph.

\item Compare this fundamental frequency $f_1$ with the frequency $f_0$ of the spring mass
system with one end fixed and the zero mass attached (as determined in Part B) and
show that $$f_0 = \frac{f_1}{2}$$
\end{enumerate}

\begin{question}
\paragraph{Question:} Can you think of why the two frequencies should be related by a factor of two? (You may use the analogy between the spring and an air column.)
\end{question}


\section*{References}

\begin{enumerate}
\itemsep0em
\item J. Christensen, \textit{Am. J. Phys}, 2004, 72(6), 818-828.

\item T. C. Heard, N. D. Newby Jr, Behavior of a Soft Spring, \textit{Am. J. Phys}, 45 (11), 1977,
pp. 1102-1106. 

\item H. C. Pradhan, B. N. Meera, Oscillations of a Spring With Non-negligible Mass, \textit{Physics Education (India)}, 13, 1996, pp. 189-193.

\item B. N. Meera, H. C. Pradhan, Experimental Study of Oscillations of a Spring with Mass Correction, \textit{Physics Education (India)}, 13, 1996, pp. 248-255.

\item Rajesh B. Khaparde, B. N. Meera, H. C. Pradhan, Study of Stationary Longitudinal Oscillations on a Soft Spring, \textit{Physics Education (India)}, 14, 1997, pp. 130-19. 

\item H. J. Pain, \textit{The Physics of Vibrations and Waves}, 2nd Ed, John Wiley \& Sons, Ltd., 1981.

\item D. Halliday, R. Resnick, J. Walker, \textit{Fundamentals of Physics}, 5th Ed, John Wiley \& Sons, Inc., 1997.

\item K. Rama Reddy, S. B. Badami, V. Balasubramanian, \textit{Oscillations and Waves}, University
Press, Hyderabad, 1994.
\end{enumerate}

\newpage
\chapter{The Three-Terminal Black Box}

\section*{Objectives}

\begin{enumerate}
\item To study a three-terminal black box and identify the circuit within it along with the values of the electronic components.

\item To understand the responses of different active and passive circuit elements and their combinations, and to learn to recognise them.

\item To exercise your deductive and inductive powers, much as real physicists must do with real experiments.
\end{enumerate}


\section*{Introduction}

You are given a sealed box with three terminals marked $A$, $B$, and $C$ which connect to a circuit inside. This circuit consists of \textbf{three} electronic components of different kinds, arranged between the three terminals in either a ``star'' or ``delta'' connection, as shown in Figure (\ref{fig:bb}). You are expected to identify each component's type, value, and orientation with respect to the terminals. 

\begin{figure}[!htb]
    \centering
    \includegraphics[scale=0.3]{figs/bb.png}
    \caption{Different configurations possible within a blackbox: it will be specified if the box contains a star (\textit{left}) or delta (\textit{right}) connection. Do not attempt to open the box! }
    \label{fig:bb}
\end{figure}

\begin{imp}
You will be told beforehand if the connection is ``star'' or ``delta'' as you cannot distinguish them otherwise. This is because, if the circuit contains only resistors, every star connection has an equivalent delta connection and vice versa.

The black box will always contain \textbf{three} electronic components, but they need not all be distinct. Some components may be repeated; a black box may be composed of (i) three resistors, (ii) two resistors and a diode, (iii) a battery and two diodes, and so on. Each arm (see Figure (\ref{fig:bb})) may contain anything from all to none of the components.
\end{imp}


\subsection*{Star and Delta Connections}

Standard 3-component circuits take on two major forms with names that represent the way in which the components are connected: a \textbf{Star} sometimes represented by Y, and a Delta connected network sometimes represented by a triangle or $\Delta$.

\begin{figure}[!htb]
       \begin{subfigure}[t]{0.3\textwidth}
				\centering
                \includegraphics[scale=0.4]{bb-comp.png}
                \captionsetup{justification=centering}
                \caption{Component or combination \\of components}
       \end{subfigure}%
       \begin{subfigure}[t]{0.3\textwidth}
				\centering
                \includegraphics[scale=0.4]{bb-star.png}
                \caption{Star connection}
        \end{subfigure}%
        \begin{subfigure}[t]{0.3\textwidth}
        		\centering
                \includegraphics[scale=0.4]{bb-delta.png}
                \caption{Delta connection}
        \end{subfigure}%
        \caption{In a star connection, one node is inaccessible and so only two components are connected at any time. In a delta connection, all the components are connected.}
        \label{fig:starAndDelta}
\end{figure}

\begin{figure}[!htb]
\centering
\includegraphics[scale=0.4]{figs/bb-starToDelta.png}
\caption{Every star connection has an \textit{equivalent} delta connection. If the components happen to be resistors, explicit formulae may be derived for each arm.}
\label{fig:starToDelta}
\end{figure}

 Both these forms, represented in Figure (\ref{fig:starAndDelta}), are indistinguishable from each other, in that for every star connection there is an equivalent delta connection (with different values for the individual components) and vice versa. It is for this reason that you must keep in mind what form circuit in the black box has before drawing any conclusions.

\begin{imp}
If all the components are resistances, the following formulae will allow you to move between a star connection and its associated delta connection and vice versa:

\begin{minipage}{0.5\textwidth}
\centering
\paragraph*{Delta to Star Network:}
\begin{equation*}
\begin{aligned}
P &= \frac{A B}{A + B + C}\\
Q &= \frac{A C}{A + B + C}\\
P &= \frac{B C}{A + B + C}\\
\end{aligned}
\end{equation*}
\end{minipage}
\begin{minipage}{0.5\textwidth}
\centering
\paragraph*{Star to Delta Network:}
\begin{equation*}
\begin{aligned}
A &= \frac{P Q + Q R + R P}{R}\\
B &= \frac{P Q + Q R + R P}{Q}\\
C &= \frac{P Q + Q R + R P}{P}\\
\end{aligned}
\end{equation*}

\end{minipage}

\end{imp}

\subsection*{Passive circuit components}

Passive circuit components are electrical components that cannot control the current in a circuit. They do not generate energy, but instead dissipate or store it.

\begin{description}
\item[Resistors] A resistor is an electronic component used to oppose or limit the current in a circuit. It's behaviour is dictated by Ohm’s law, which states that the voltage applied across the terminals of a resistor is directly proportional to the current flowing through it, with the constant of proportionality called the \textit{resistance}, $V=IR$.

% \begin{equation*}
% V = I R
% \end{equation*}

\item[Capacitors]

A capacitor, made from two conductive plates with an insulator between them, stores electrical energy in the form of an electric field. It blocks DC signals (when fully charged) and allows the AC signals to pass through it. The charge stored in a capacitor is given by $Q=CV$.

% \begin{equation*}
% Q = CV
% \end{equation*}

When used with a resistor, the time a capacitor takes to charge or discharge is measured in units of an intrinsic time scale, known as the time constant $\tau = RC$ of the circuit.

\end{description}



\subsection*{Active circuit components}
An active device is any type of circuit component with the ability to electrically control electron flow in the circuit.

\begin{description}
\item[Batteries]

Charges can be separated to produce a voltage. A battery uses a chemical reaction to produce energy and separate opposite sign charges onto its two terminals. As the charge is drawn off by an external circuit, doing work and finally returning to the opposite terminal, more chemicals in the battery react to restore the charge difference and the voltage.


\item[p-n junction Diodes]

These are two-terminal semiconductor devices, which allow the electric current in only one direction while blocking it in the reverse direction. A diode has an anode (or positive end) containing positive charge carriers called ``holes'', and a cathode (or negative end) containing negative charge carriers (electrons). The interface between these ends forms a region without any charge carriers called the \textbf{depletion layer}. 

The process of applying the external voltage to a p-n junction semiconductor diode is called biasing. If the diode is \textbf{forward biased} -- that is, when a positive voltage is applied to the anode and a negative voltage to the cathode -- it allows the charge carriers, and hence current, to flow. On the other hand, if the diode is \textbf{reverse biased} -- negative to the anode and positive to the cathode -- it blocks the current flow. In the conventional symbol for a diode, the arrowhead indicates the conventional direction of electric current when the diode is forward biased.

\begin{figure}[!htb]
\centering
\includegraphics[scale=0.4]{figs/diode_characteristics.png}
\caption{A diode and its current-voltage characteristics: for small (positive) values of input voltage, the output current varies exponentially.}
\label{fig:diodeChar}
\end{figure}

Unlike a resistor, a diode does not behave linearly with respect to the applied voltage as the diode has an exponential current-voltage (I-V) relationship and therefore cannot be described by simply using an equation such as Ohm’s law. The reason for this is that the width of the depletion layer decreases with increase in positive voltage. After a certain ``knee'' voltage (0.3 V for Germanium semiconductors and 0.7 V for Silicon semiconductors) the width of the barrier effectively goes to zero, and the diode acts like a short circuit with zero resistance. When a junction diode is reverse biased the thickness of the depletion region increases and the diode acts like an open circuit blocking any current flow (except for a very small leakage current), until an ``avalanche'' voltage when the device undergoes a breakdown and gets shorted, leading to maximum current flow in the \textit{opposite} direction.


\end{description}




\section*{Experimental Setup}

\subsection*{Apparatus}

\begin{enumerate}
\item A black box with three connecting terminals marked $A$, $B$, and $C$
\item A variable DC power supply (\textit{Keltronix})
\item Two digital multimeters (\textit{MECO 603})
\item A signal generator (\textit{Testronix 72})
\item A Digital Storage Oscilloscope (\textit{GW INSTEK–GDS-1102U})
\item A digital stopwatch (\textit{Racer})
\item A resistor ladder
\item Connecting wires

\end{enumerate}

\subsection*{Description}

\begin{description}
\item[Digital Multimeter (\textit{MECO 603})]

A multimeter is an instrument used to measure multiple parameters like voltage, current, and resistance. In this experiment you will use the given digital multimeters (Model \textit{MECO - 603}) \textbf{\textit{only}} for the measurement of DC and AC voltage and current. You will have to use input sockets marked COM, V/$\Omega$ and mA (or 20A) for the required measurements. Note that there are two input sockets marked mA and 20A for the current measurement. The socket marked mA may be used for measuring current below 250 mA and socket marked 20A may be used for measuring current up to 20 A. You will have to select the appropriate function and the range using the rotary switch provided on the multimeter. The value of voltage or current is displayed on the LCD screen. The multimeter is turned off by turning the dial to the appropriate setting.

\item[Variable DC Power Supply (\textit{Keltronix})]

The DC Variable Power Supply can either act as a source of constant voltage (CV), or constant current (CC), indicated by the two LEDs present on it. We will be using it as a constant voltage source, so make sure the CV LED is lit. An ideal voltage source is one that produces a fixed voltage irrespective of the current the load (your circuit) demands of it. Of course, this is not realistic; the supply given to you can supply a maximum potential difference of 15V, and a maximum current of 1A.

The voltage and the current in the circuit can be changed using the three knobs: voltage coarse, voltage fine, and current. The given power supply also displays the values of the output voltage and current. Do not use these values since the multimeter will be more reliable. 

If the current knob is set to maximum, the supply acts as a constant voltage source. The constant voltage it supplies can be varied using the voltage knobs, and is independent of the load (your circuit) attached to it.\footnote{Similarly, if the voltage knob is set to maximum, the supply acts as a constant current source.}

\begin{imp}
If the current your circuit draws at any voltage is beyond 1A, the power supply will not be able to supply it. This could happen if you short the two terminals of the power supply (don't!) or if you are close to the knee voltage of a diode (as in this case, the circuit has effectively zero resistance). In this case, the power supply's LED will shift from CV to CC, indicating that it is providing the maximum current possible. Avoid this situation, as it could damage the power supply.
\end{imp}

\item[Signal Generator (\textit{Testronix 72})]

A signal generator is used to generate simple repetitive waveforms in the form of an alternating electrical wave. Typically, it will produce simple waveforms like sine, square, and triangular waves, and will allow you to adjust the frequency and amplitude of these signals. The instrument given to you generates sine and square waveforms. The output may be taken from the respective output sockets. The frequency can be adjusted by turning the dial and selecting the appropriate Frequency Multiplier button. For example, turning the dial to 3 and selecting the 10k button would provide an output waveform with a frequency of 30kHz. Similarly the amplitude switch, used in conjunction with the appropriate buttons, can be used to adjust the amplitude of the output waveform from 0V to 30V.

\end{description}




\subsection*{Precautions}

\begin{itemize}
\item The use of a multimeter to measure the resistance is strictly prohibited.
\item Do not try to open the black box, use only terminals $A$, $B$, and $C$ for the connections.
\item Passing more current through the multimeter's (250 mA or 20A) sockets than they can handle will cause its fuses to burn out, leading to an open circuit. Use an appropriate resistance to limit the current in the circuit. The current should not exceed 500 mA.
\item \textbf{Do not} short the outputs of the power supply, it could damage the equipment.

\item When drawing circuit diagrams use the standard symbols.
\item The digital multimeter, the DC power supply or the signal generator should be turned off if not in use.

\end{itemize}


\section*{Procedure}

\subsection*{Part A}

The black box will contain only combinations of resistors, p-n junction diodes, and batteries in a \textit{\textbf{star}} connection.

\begin{question}
    \paragraph{Question:} Since not all of the elements will necessarily be present in a box, which is the first element you will test for?
    \end{question}

\begin{enumerate}
    \item Choose the first element that you would like to detect (or eliminate), and design a method to do this.
    
    
    \item Repeat this for all the components.
    
    \begin{question}
    \paragraph{Question:} How could you tell which way a diode was oriented? 

    \paragraph{Question:} How would you differentiate between a diode and a resistor?
\end{question}
    
    \item Give your answer in the form of a circuit diagram showing the terminals $A$, $B$, and $C$ clearly. 
\end{enumerate}

\begin{tip}
You will need to give \textit{all} the information required in order to reconstruct the circuit. For example, you will need to determine the values of the resistors, the orientation of the diodes, and the voltage and orientation of the battery, if any of the above are used in the circuit.
\end{tip}

\begin{imp}
Clearly report the procedural steps you have taken with reference to (i) your complete plan of experiment, (ii) the data you will collect, (iii) how the collected data will be analysed and (iv) how your analysis will be interpreted. Your reporting should be comprehensive, and all important steps should be mentioned in your answer sheet.
\end{imp}

\subsection*{Parts B, C, \& D \textit{(optional)}}

\begin{itemize}
    \item \textbf{Part B:} The black box will contain only combinations of resistors, p-n junction diodes and capacitors in a \textit{\textbf{star}} connection. 
    
    \item \textbf{Part C:} The black box will contain only combinations of resistors, p-n junction diodes, and batteries in a \textit{\textbf{delta}} connection. 
    
    \item \textbf{Part D:} The black box will contain only combinations of  resistors, capacitors, and inductors in a \textit{\textbf{star}} connection.
\end{itemize}

% \subsection*{Part C \textit{(optional)}}
% The black box will contain only combinations of resistors, p-n junction diodes, and batteries in a \textit{delta} connection. 

% \subsection*{Part D \textit{(optional)}}
% The black box will contain only combinations of  resistors, capacitors, and inductors in a \textit{star} connection.


\section*{References}

\begin{enumerate}

\item \href{https://www.elprocus.com/major-electronic-components/}{Overview of Various Basic Electronics Components.} Retrieved June 28, 2019, from (\nolinkurl{https://www.elprocus.com/major-electronic-components/})

\item \href{https://www.electronics-tutorials.ws/dccircuits/dcp_10.html}{Star Delta Transformation.} Retrieved June 28, 2019, from \nolinkurl{https://www.electronics-tutorials.ws/dccircuits/dcp_10.html} 

\item \href{https://www.electronics-tutorials.ws/diode/diode_3.html}{Electronics Tutorials: PN Junction Diode.} Retrieved June 28, 2019, from \nolinkurl{https://www.electronics-tutorials.ws/diode/diode_3.html}

\end{enumerate}


\newpage
% \title{The Incandescent Lamp and the Inverse Square Law}
% \author{An Introduction to Physics through Experiments}
% \date{}
% \maketitle

\chapter{The Incandescent Lamp and the Inverse Square Law}

\section*{Objectives}

\begin{enumerate}
\item To understand the current-voltage characteristics of an incandescent lamp's filament.
\item To understand the dependence of the lamp's power on the resistance of its filament.
\item To observe and understand the variation of intensity with distance for a point source.
\end{enumerate}




\section*{Introduction}

Thermal radiation is the conversion of thermal energy into electromagnetic energy. You should remember from earlier courses that the \textbf{temperature} of an object is a statistical property that -- at microscopic scales -- is a reflection of the average kinetic energy of the atoms or molecules present within that object. Thus, an increase in temperature would cause these atoms or molecules to start `jiggling' about more erratically (at least on an average). The higher the temperature, the more rapid this motion. This irregular thermal motion causes the charges that comprise the atoms and molecules to oscillate. 

A famous result in electrodynamics states that accelerating charges radiate electromagnetic radiation. Each oscillation at a particular frequency can be considered a tiny ``antenna'' that emits and receives electromagnetic radiation. Of course, since these oscillations are due to random collisions within the object, there is a \textit{continuous range} of possible frequencies of oscillations and thus of the emitted electromagnetic radiation. An object that emits radiation over a wide range of frequencies is known as a \textit{black body}, and the characteristic spectrum of radiation it produces is called black-body radiation. For example, as a piece of iron is heated to increasingly high temperatures, it first glows red, then yellow, and finally white. In short, all the colours of the visible spectrum are represented. In fact, the sensation of heat you get from the iron even before it begins to glow red is an indication of the emission of infrared radiation.

% \todo[inline]{Not happy with the definition of a blackbody...}


The spectrum of a blackbody was known quite well experimentally, however no theoretical approach could describe it accurately. There were two different laws, each of which fit the experimental data in different regimes, but neither of which explained the full spectrum. For low frequencies (or long wavelengths), the law was the ``Rayleigh-Jeans law'', while for high frequencies (or short wavelengths) it was known as ``Wein's displacement law''.

\begin{figure}[!htb]
\centering
\begin{subfigure}[b]{0.45\textwidth}
        \includegraphics[width=\textwidth]{planck.png}
        \caption{Classical approximations}
        \label{planck_approx}
    \end{subfigure}%
    \begin{subfigure}[b]{0.45\textwidth}
        \hspace{0.5cm}\includegraphics[width=0.8\textwidth]{planck_temp.jpg}
        \caption{Variation with temperature}
        \label{planck_temp}
    \end{subfigure}
\caption{The Planck Spectrum for Blackbodies}
\label{planck}
\end{figure}

The problem was solved by the German physicist Max Planck who realised that classical physics could not be used for such blackbody radiation, and thereby ushered in Quantum Mechanics by postulating that the energies were absorbed and emitted in specific quanta. The resulting spectrum was found to be characterised by one single parameter: the temperature of the object, and was found to fit a remarkable number of spectra from the sun to the Cosmic Microwave Background Radiation left as a remnant from the Big Bang.

\begin{question}
\paragraph{Question: } It is often said that the Cosmic Microwave Background Radiation has a ``temperature'' of 2.73 K. What do you think this means?
\end{question}


An incandescent lamp is composed of a wire filament protected from oxidation with a glass or fused quartz bulb that is filled with inert gas or a vacuum. If an electric current is passed through such a wire, it will begin to get heated. If the wire is heated to a sufficiently high temperature, it will give off light in the visible regime.  This is the basis of a resistive incandescent lamp. The high melting point of tungsten (3680 K) and its low vaporisation pressure makes it a suitable material to be used as the filament of almost all incandescent lamps. The brightness and the overall colour of emitted light depends on the temperature of the filament for a given lamp, which is a non-linear resistive element.

At the heart of the lamp is a resistive element, which should follow Ohm's Law. Thus, when a voltage is applied across the lamp, a current flows through it, and the two are related by 

\begin{equation}
    V = R I.
    \label{ohm}
\end{equation}

However, as more current flows through the the filament, it gets heated which in turn changes its resistance. Thus, a more appropriate way of writing the above equation for the filament is 

\begin{equation}
    V = R(I) I
\end{equation}

where $R$, despite being an ``instantaneous'' resistance, is not a constant value. Suppose we don't know the exact form of $R(I)$. One assumption might be to assume some sort of power law behaviour. Thus,

\begin{equation}
    V = K I^a
    \label{isl-Vpowerlaw}
\end{equation}

where $K$ and $a$ are constants, and $V$ is the voltage across the lamp and $I$ the current passing through it.

% Let us consider how the resistance varies as a function of temperature $T$. It is reasonable to assume that the temperature dependence of the resistance of the filament is given by

% \begin{equation}
%     R = R_0 \left( 1 + \alpha (T-T_0) \right)
% \end{equation}

% where $R_0$ is the resistance of the filament at the ambient temperature $T_0$, and $T$ is the temperature of the filament, and $\alpha$ is known as the temperature coefficient of resistance of the material.

% {\color{red}So this is my problem: I want to find the relation between temperature $T$ and current. I know the temperature $\Delta T \propto E_\text{rad} = m c \Delta T$, and the energy is $\int P \dd t$. 

% Now I need the power, and it's simply $VI$. I don't use $I^2R$ or $V^2/R$ since it doesn't give the true dependence on $I$, since $R(I)$. But since $R$ changes with time, and therefore so does $I$. Even if I assume the energy is simply $VIt$, this doesn't remove the problem, as now 

% \begin{equation}
%     \Delta T \propto VI
% \end{equation}

% Thus $R$ seems to depend on $V$ as well as $I$, and at this point I feel I'm doing something wrong.

% Alternatively, I could do the following: I say the energy (and hence $\Delta T$) is $\propto I$, and therefore $R(I) = R_0 (1 + \kappa I)$, which means that 

% \begin{equation}
%     V = R_0 (I + \kappa I^2)
% \end{equation}
% } 


% \todo[inline]{I'm increasingly coming to the conclusion that this entire experiment needs a rethink....}

Similarly, the power in the circuit can be calculated by the formula: 

\begin{equation}
    P = V I.
\end{equation}

For a normal resistor which follows Equation (\ref{ohm}) (known as Ohm's law), we could rewrite the equation as 

\begin{equation*}
    P = VI = I^2 R = \frac{V^2}{R},
\end{equation*}

however, in the case of an incandescent lamp where the ratio between the current and voltage is not a constant, this is no longer necessarily true. One finds for the resistive lamp, the empirical relation
\begin{equation*}
P = C R^n
\end{equation*}

where $P$ is the power supplied to the lamp, $R$ is the resistance of the filament of the lamp, and $n$ and $C$ are constants whose values depend on the material used for the filament of the lamp. The above relation yields better results when the temperature of the filament of the lamp is approximately above 1800 K for the given lamp. For such temperatures the resistance R of the lamp is found to be directly proportional to the absolute temperature T of the filament of the lamp.



\subsection*{Point sources}


\begin{minipage}{0.5\linewidth}
%\vspace{-1.5cm}
A \textbf{point source} is a single identifiable localised source of, say, light. Such a source has negligible extent, distinguishing it from other source geometries, and are called point sources because in mathematical modelling, they can usually be approximated as a mathematical point to simplify analysis. The actual source need not be physically small, if its size is negligible relative to other length scales in the problem. For example, in astronomy, stars are routinely treated as point sources, even though they are in actuality much larger than the Earth.

For light or sound waves, a point source radiates the same intensity of radiation in all directions. That is, it has no preferred direction of radiation and radiates uniformly in all directions over a sphere centred on the source.
\end{minipage}
\begin{minipage}{0.5\linewidth}
\centering
\includegraphics[scale=0.7]{isl.png}
\end{minipage}



\section*{Experimental Setup}

\subsection*{Apparatus}

\begin{enumerate}
\item A DC variable power supply (\textit{Optochem})
\item Two digital multimeters with probes (\textit{MECO 603} and \textit{Victor VC97})
\item A 12 V, 21W incandescent lamp
\item A stand to hold the lamp
\item A Resistor Ladder
\item Connecting cords
\item A convex lens
\item A photodiode

\end{enumerate}


\subsection*{Description}

\begin{description}
\item[Digital Multimeter (\textit{MECO 603}, \textit{Victor VC97})]

A multimeter is an instrument used to measure multiple parameters like voltage, current, and resistance. In this experiment you are given two digital multimeters (Models \textit{MECO - 603} and \textit{Victor VC97}).\footnote{The Victor multimeter has a greater current sensitivity.} You will have to use input sockets marked COM, V/$\Omega$ and mA (or A or 20A) for the required measurements. Note that there are two input sockets marked mA and 20A (10A for the Victor multimeter) for the current measurement. The socket marked mA may be used for measuring current below 250 mA and socket marked 20A (or 10A) may be used for measuring current up to 20 A (or 10A). You will have to select the appropriate function and the range using the rotary switch provided on the multimeter. The value of voltage or current is displayed on the LCD screen. The multimeter is turned off by turning the dial to the appropriate setting.

\item[Variable DC Power Supply (\textit{Optochem})]

The DC Variable Power Supply can either act as a source of constant voltage (CV), or constant current (CC). An ideal voltage source is one that produces a fixed voltage irrespective of the current the load (your circuit) demands of it. Of course, this is not realistic; the supply given to you can supply a maximum potential difference of 15V, and a maximum current of 2A.

The voltage and the current in the circuit can be changed using the three knobs: voltage coarse, voltage fine, and current. The given power supply also displays the values of the output voltage and current. Do not use these values since the multimeter will be more reliable. 

If the current knob is set to maximum, the supply acts as a constant voltage source. The constant voltage it supplies can be varied using the voltage knobs, and is independent of the load (your circuit) attached to it.\footnote{Similarly, if the voltage knob is set to maximum, the supply acts as a constant current source.}

\begin{imp}
If the current your circuit draws at any voltage is beyond 2A, the power supply will not be able to supply it. This could happen if you short the two terminals of the power supply (don't!) or connect it in any other way to a circuit that has effectively zero resistance. In this case, the power supply's output current will be 2A, indicating that it is providing the maximum current possible. Avoid this situation, as it could damage the power supply.
\end{imp}

\item[Photodiode]

A photodiode is a semiconductor device that converts light into an electrical current. The current is generated when photons of sufficient energy are absorbed in the photodiode, creating an electron-hole pair through the photoelectric effect.

\end{description}


\subsection*{Precautions}

\begin{itemize}
\item Do not apply a voltage more than 12 V to the lamp.
\item Clamp the lamp carefully on the retort stand so that the wires do not get shorted accidentally.
\item Use the digital multimeter for the measurement of DC voltage and current.  Choose the appropriate range for the measurement.
\item The incandescent lamp should not be kept connected to the power supply for long duration.
\item Passing more current through the multimeter's (250mA or 20A) sockets than they can handle will cause the fuses in the multimeter to burn out, leading to an open circuit. Use an appropriate resistance to limit the current in the circuit.
\item When drawing circuit diagrams use the standard symbols.
\item The digital multimeter and the DC power supply should be turned off if not in use.
\end{itemize}


\section*{Procedure}

\subsection*{Part A}

In this part, you will design an experiment to study the VI characteristics of an incandescent lamp. 

\begin{enumerate}
    \item  Begin by drawing the necessary circuit diagram, and connect the lamp to the power supply through a multimeter working as an ammeter. 
    
    \begin{imp} 
        Make sure you know how much current is flowing through the circuit so you know whether to use the 250mA or the 20A plugs on the multimeter. 
    \end{imp}
    
    \begin{question}
        \paragraph{Question:} How would you find the maximum resistance of a 12V, 21W lamp?
    \end{question}
    
    \item Connect a multimeter as a voltmeter across the lamp.
    
    \item Vary the voltage across the lamp and measure the current.
    
    \item Plot an appropriate graph to determine $K$ and $a$ in Equation (\ref{isl-Vpowerlaw}).
\end{enumerate}


\begin{question}
\paragraph{Question:} Describe qualitatively how you would expect the resistance of the lamp to vary with the current. Explain why you think this is the case.
\end{question}

\subsection*{Part B}

\begin{enumerate}
    \item Design an appropriate procedure to determine the value of $n$. You may use the  data collected in \textbf{Part A}, or you may perform new measurements.
    
    \item Plot an appropriate graph to determine $C$ and $n$.
\end{enumerate}


\subsection*{Part C}

In this part, you will design an experiment to create a point source and then study the variation of intensity of the light with distance.  For this you have to use the given photodiode with a digital multimeter to measure the intensity of light. We assume that the relation between the intensity of light incident on the sensing area of the photodiode and its output current is linear. 

\begin{enumerate}
    \item Use the converging lens to create a point source. You may use the screen with a hole in it to allow the passage of a small amount of light of this ``point source'' that may then be detected by the photodiode. The screen is now your point source. 
    
    \item Connect the output of the power supply to the given lamp and adjust the voltage applied to the lamp to be around 12 V. 
    
    \item Keep the photodiode initially as close as possible to the screen acting as your source. 
    
    \item Study the variation of current in the photodiode with the distance $d$ between the source and the photodiode. (You have to take readings for the value of $d$ varying from around 7.0 cm to around 50.0 cm.) You may have to correct your readings to account for the ambient light falling on the photodiode. 
    \item Plot an appropriate graph to show the variation of the intensity of light with distance $d$.
    
    \begin{question}
    \paragraph{Question:} What type of graph would be the most efficient to determine the variation of the intensity of light with distance? Why? 
    
    \paragraph{Question:} After plotting this graph, you may find that points below a certain value (say, 20.0 cm) do not behave as you would expect them to. Can you explain why this is the case?
    \end{question}
\end{enumerate}



\section*{References}

\begin{enumerate}

\item Justel, T. \href{https://www.fh-muenster.de/ciw/downloads/personal/juestel/juestel/4-InkohaerenteLichtquellen-Glueh-_und_Halogenlampen_english_-1.pdf}{Incandescent and Halogen Lamps}. Retrieved from \url{https://www.fh-muenster.de/ciw/downloads/personal/juestel/juestel/4-InkohaerenteLichtquellen-Glueh-_und_Halogenlampen_english_-1.pdf}.

\item \href{https://www.elprocus.com/photodiode-working-principle-applications/}{Photodiode Working Principle, Characteristics and Applications}, \textit{ElProCus}. Retrieved from \url{https://www.elprocus.com/photodiode-working-principle-applications/}.

\end{enumerate}



\newpage
% \title{Diffraction of Light}
% \author{An Introduction to Physics through Experiments}
% \date{}
% \maketitle

\chapter{Diffraction of Light}

\section*{Objectives}

\begin{enumerate}
    \item To determine the wavelength $\lambda$ of the light emitted by a laser source by studying the diffraction of light due to plane diffraction gratings.
    \item To determine the width of the given single slit by studying its diffraction pattern.
    \item To determine the diameter of a given wire by studying its diffraction pattern.
    \item To determine the size of the circular aperture by studying its diffraction pattern.
\end{enumerate}



\section*{Apparatus}

\begin{enumerate}
    \item A 10 mW semiconductor red laser source.
    \item A 5 mW DPSS green laser source.
    \item A set of necessary mounts.
    \item A set of plane diffraction gratings of different grating spacings.
    \item A single helix (spring) set in a holder
    \item A double helix set in a holder
    \item Measuring tapes.
    \item A holder for grating.
    \item A set of screens.  
    \item A single slit of fixed width mounted on a slide.
    \item Two multiple slits mounted on slides.
    \item Circular apertures mounted on slides.
    \item A spirit level
\end{enumerate}

\section*{Introduction}
	 
In \textit{Opticks} (1704), Issac Newton wrote, ``Light is never known to follow crooked passages nor to bend into the shadow''. He explained this by describing how particles of light always travel in a straight line, and how objects kept in the path of the light cast a shadow because the particles can never spread out behind the object. However, a set of experiments on the propagation of light through small apertures performed by Francesco Grimaldi, Augustine Fresnel, Thomas Young and a few others firmly established that light actually enters into the shadow region with a definite pattern when it passes through around an edge. The resulting pattern depends on the relative size of the aperture or obstacle and the wavelength of light. If the size is much larger than the wavelength, the bending will be almost unnoticeable. However, if the two are similar in size, the diffraction will be considerable.    
     
In this experimental problem, we will use a low power solid-state laser as a source of an intense beam of monochromatic light. When light from a distant source (or a laser source) passes around a thin aperture or through a narrow aperture and is then intercepted by a viewing screen, the light produces a pattern on the screen called a \textit{diffraction pattern}. When such a beam is incident on various diffracting components like a plane diffraction grating, a single slit, a wire mesh or a two-dimensional diffraction grating, the light emerging from these components show a variety of interesting diffraction patterns. This pattern consists regions of maximum and minimum intensities, which characterise the diffracting object. 

\section*{Theory}


\begin{imp}

\begin{center}
    \textbf{Babinet's Principle}
\end{center}

Two objects are called \textit{complementary} if one of them is transparent where the other is opaque and opaque where the other is transparent. An example of this is given in Figure (\ref{fig:complementary}).~\\

\textbf{Babinet's principle} states that

\begin{center}
``Complementary objects \textbf{produce the same diffraction pattern}, except for the intensity of the central maxima.''
\end{center}

This is a highly non-intuitive result, and one that you may study in some of your later courses. The mathematics of this are far beyond the scope of this lab session. 
Henceforth we will not differentiate between the patterns made by such complementary objects. We will thus speak of a single slit (or of a thin wire) having the `same' diffraction pattern, keeping in mind that the intensities of the central maxima are different.
\end{imp}

\begin{figure}[!htb]
    \centering
    \includegraphics[scale=0.6]{figs/complementary.png}
    \caption{Transparency of a thin wire and of a single slit.}
    \label{fig:complementary}
\end{figure}

\begin{question}
\paragraph{Question:} What are the complementary objects of the following:

\begin{enumerate}
\item A transparent sheet of paper.
\item A circular aperture of diameter $d$.
\end{enumerate}
\end{question}



\subsection*{The Single Slit (or Thin Wire)}

When a (monochromatic) beam of light such as a laser is incident on a narrow single slit, the light emerging from the slit shows a diffraction pattern on a screen. The distribution of the intensity of light received on a screen show a pattern of varying intensity consisting of a bright central maximum with alternate minima and maxima of decreasing intensity on either side, known as a \textit{Fraunhofer diffraction pattern}.

Similarly, instead of a slit of width $a$, supposing we had a \textit{wire} of diameter $a$ placed as an obstruction to the laser beam. The resulting intensity pattern as observed on a screen almost exactly the same as that observed for a single slit. This intensity distribution can be written as a function of the angle $\theta$ as

\begin{equation*}
    I(\theta) = I(0) \left( \frac{\sin \beta}{\beta} \right)^2, \quad \quad  \beta = \frac{\pi a \sin \theta}{\lambda}
\end{equation*}

This function has been represented in Figure (\ref{fig:singleslit}).

\begin{figure}[!htb]
    \centering
    \includegraphics[width=0.75\textwidth]{figs/singleslit.png}
    \caption{The diffraction pattern produced by a single-slit aperture (or thin wire).}
    \label{fig:singleslit}
\end{figure}


Moving away from the central spot, when $\sin\beta=0$ (but $\beta \neq 0$), the intensity vanishes! The positions of these \textbf{minima} (zeros) of the intensity distribution pattern are given by the relation 

\begin{equation}
   a \sin{\theta_m} = \pm m \lambda  \quad\quad\quad \text{for    m  = 1, 2, 3,}\hdots,
   \label{singleslit}
\end{equation}

where $\lambda$ is the wavelength of the incident light, $a$ is the width of the slit and $\theta_m$ is the angle corresponding to $m$th minimum. Here the $\pm$ refers to either side of the central spot ($\theta = 0$).


\subsection*{The Double Slit}

We can now imagine two single slits of the same width next to each other. Such an arrangement is called a double-slit aperture. We could think of this aperture (shown in Figure (\ref{fig:doubleslit})) as a combination of a single slit of width $d$, and a thin wire of width $a$. 

This situation is slightly more complicated than the previous one; the thin obstacle of width $a$ will produce a diffraction pattern like the one in the previous section. However, the light from the two slits on either side will \textbf{interfere}, producing an overall interference pattern which also has its own maxima and minima! The mathematics of this is too difficult to understand right now; you will cover it in a later laboratory.

The resultant intensity distribution is given by   

\begin{equation}
    I(\theta) = I(0) \cos^2\delta \left( \frac{\sin \beta}{\beta} \right)^2, \quad \quad  \beta = \frac{\pi a \sin \theta}{\lambda}, \quad \delta = \frac{\pi d \sin \theta}{\lambda}
\end{equation}

For a screen placed at a large distance $D$ from the wire, the positions of the minima on the screen are observed at 

\begin{equation}
    \begin{aligned}
        x_{\pm n} = \pm n \frac{\lambda D}{a}, &\quad& \text{due to diffraction},\\
        x_{\pm m} = \pm \left( m - \frac{1}{2}\right) \frac{\lambda D}{d}, &\quad& \text{due to interference}
    \end{aligned}
    \label{doubleslit}
\end{equation}


\begin{figure}[!htb]
    \centering
    \includegraphics[width=0.75\textwidth]{figs/doubleslit.png}
    \caption{The diffraction pattern produced by a double-slit aperture is a combination of two patterns (diffraction due to a single wire and interference due to two slits).}
    \label{fig:doubleslit}
\end{figure}



\begin{question}
\paragraph{Question:} Look at the above condition for the fringes produced due to diffraction in Equation (\ref{doubleslit}). Is it different from the result obtained in the previous section (Equation (\ref{singleslit}))? If yes, why? If no, why not?~\\
\paragraph{Question:} What is the ``complementary'' object to a double-slit?
\end{question}



\subsection*{Plane Diffraction Grating}

A transmission diffraction grating consists of a large number of slits separated from one another by an opaque region. The grating concentrates the diffracted light along a particular direction in contrast to the single slit, which has a rather broad diffraction maximum. The maxima (bright intense spots) produced by a grating are usually called the principal maxima. They are quite intense and are also widely separated; what cannot be detected visually are the large number of secondary maxima which lie between neighbouring principal maxima.

The expression, relating wavelength $\lambda$ of light used and the grating spacing $d$, with angle of deviation $\theta$ is
\begin{equation*}
    d \sin{\theta_m} = m \lambda,  \quad\quad\quad \text{for    m  = 1, 2, 3,}\hdots
\end{equation*}

In the above expression, $m$ represents the order of \textbf{maxima} points and the angle $\theta_m$  corresponds to  $m$th  order maximum intensity point. This relation is valid for a single slit and for the wire-like obstacle.

\begin{figure}[!htb]
    \centering
    \includegraphics[width=0.75\textwidth]{figs/grating.png}
    \caption{The diffraction pattern produced by a plane diffraction grating.}
    \label{fig:grating}
\end{figure}


\subsection*{The Circular Aperture}

The diffraction pattern due to a circular aperture (known as an \textit{Airy diffraction pattern}) is similar to a single slit diffraction but the mathematics involved is more complicated which gives the expression nearly identical to that of the single slit. Hence we may apply the same expression to the diffraction due to a circular aperture, 

\begin{equation*}
    d \sin{\theta_m} = \overline{m} \lambda,  \quad\quad\quad \text{for    m  = 1, 2, 3,}\hdots
\end{equation*}

where $d$ is the diameter of the circular aperture and $\theta_m$ is the angle of deviation for the $m$th dark ring. The variable $\overline{m}$ has the following values:

\begin{equation*}
    \begin{aligned}
        m = 1 &\quad& \overline{m}=1.22\\
        m = 2 &\quad& \overline{m}=2.23\\
        m = 3 &\quad& \overline{m}=3.23\\
        m = 4 &\quad& \overline{m}=4.24\\
    \end{aligned}
\end{equation*}

\subsection*{Single and double helices:}
  
\begin{figure}[!htb]
    \centering
    \begin{subfigure}[b]{0.45\textwidth}
                \centering
                \includegraphics[width=0.75\textwidth]{figs/singlehelix.png}
                \caption{Schematic of a single helix (like RNA)}
                \label{fig:singlehelix1}
        \end{subfigure}%
        \begin{subfigure}[b]{0.45\textwidth}
                \centering
                \includegraphics[width=0.75\textwidth]{figs/doublehelix.png}
                \caption{Schematic of a double helix (like DNA)}
                \label{fig:doublehelix1}
        \end{subfigure}
\end{figure}

Now consider a set of four identical wires, the net intensity distribution is a combination of diffraction from each wire and interference due to pairs of wires and hence depends on $a$, $d$ and $s$ (see Figure \ref{fig:fourwires}).

In other words, the combination of three different intensity patterns is observed.

\begin{figure}[!htb]
    \centering
    \begin{subfigure}[b]{0.45\textwidth}
                \centering
                \includegraphics[width=0.75\textwidth]{figs/fourwires.png}
                \caption{Projection of four identical wires}
                \label{fig:fourwires}
        \end{subfigure}%
        \begin{subfigure}[b]{0.45\textwidth}
                \centering
                \includegraphics[width=0.75\textwidth]{figs/doublehelix-schema.png}
                \caption{The double helix given in the sample.}
                \label{fig:doublehelix2}
        \end{subfigure}
        \caption{Diffraction patterns of obstacles with three length scales.}
\end{figure}

\section*{Description}

In \textbf{Part A}, you will determine the wavelength of light of a laser using a set of diffraction gratings.

In \textbf{Part B}, you will use the laser whose wavelength you have just determined to measure the width of a single slit.

In \textbf{Part C}, you will study diffraction pattern of a circular aperture.

In \textbf{Part D (\textit{optional})}, you will study the diffraction pattern of a single helix.

In \textbf{Part E (\textit{optional})}, you will study the diffraction pattern of a double helix.


\section*{Procedural Instructions}

\subsection*{Part A}

Observe effect of colour and also of white light on the diffraction pattern obtained by a suitable grating. Then choose an appropriate diffraction grating and perform the measurements to determine the wavelength $\lambda$ of the laser. 

\begin{question}
\paragraph{Question:} Estimate the error in the value of the wavelength of light.~\\

\paragraph{Question:} What are the sources of error in the above-determined value of $\lambda$? What measures should be taken to minimise these errors? ~\\

\paragraph{Question:} Tilt the grating at an angle. How does this affect the diffraction pattern?
\end{question}

Now repeat this procedure for different gratings (with different values of $d$), and calculate wavelength $\lambda$ as accurately as possible.

\subsection*{Part B}

Design and perform the necessary experiment with a single slit of fixed width and determine the width d of the given single slit.

\begin{question}
\paragraph{Question:} Tilt the slit at an angle. How does this affect the diffraction pattern?
\end{question}


\subsection*{Part C}

Now take the given circular aperture as the diffracting object and determine the diameter of the circular aperture.

\subsection*{Part D}

Study of the diffraction pattern due to a helical spring and determine pitch of the spring and thickness of its wire. 


\begin{question}
\paragraph{Question:} Can you explain the form of the diffraction pattern observed? Does this part of the experiment relate in any way to Part B? 
\end{question}

\subsection*{Part E}

Study of the diffraction pattern due to a double helix (as in our DNA) and determine all its parameters, as shown in Figure (\ref{fig:doublehelix2}).

\begin{question}
\paragraph{Question} Describe the diffraction pattern obtained if you use a laser source to illuminate
\begin{enumerate}[label=(\alph*)]
    \itemsep0em
    \item A fine wire mesh,
    \item A square aperture,
    \item A rectangular aperture.
\end{enumerate}
\end{question}



\subsection*{Precautions}

\begin{enumerate}
    \item Never look directly at a laser beam with the naked eye. It may damage the eye permanently.
    \item Never point the laser at anyone else, for the same reason.
    \item Never point an optical device at a laser beam. It could damage the internal sensors.
    \item Never place highly reflective objects (such as rings, watches, and glassware) in the path of the laser beam.
    \item For proper working of laser, it should be kept on throughout. Do not put it off until you complete all your readings, but if you do not need the laser beam for measurements or alignment, use a light-blocking screen to block the  beam.
    \item Treat the laser source as you would any other electrical device: It should never be tampered with while the power cord is connected.
\end{enumerate}


\section*{References}

\begin{enumerate}
\itemsep0em
    \item Eric Stanley, Am. J. Phys., Vol.- 54, No.-10, October 1986, pp. 952.
    \item F.A. Jenkins, H.E. White, \textit{Fundamentals of Optics}, Third Edition, Mc-Graw Hill Kogakusha Ltd., Toyko, Japan, 1957, pp. 288-309, 328-350.
    \item F.W. Sears, \textit{Optics}, Third Edition, Asia Publishing House, 1958, pp 221-252.
    \item R.W. Ditchburn, \textit{Light}, Second Edition, The English Language Book Society and Blackie \& Son Ltd., 1963, pp 162-237.
    \item John Beynon, \textit{Introductory University Optics}, Prentice-Hall of India Pvt. Ltd., New Delhi (India), 1998, pp 158-190.
    \item Rajpal S. Sirohi, \textit{Wave Optics and its Applications}, Orient Longman Limited, (India), 1993, pp 169-210.
\end{enumerate}



\newpage

\vbox
{\textcolor{Blue}{\part{Lecture Demonstrations}}
\tikz[remember picture,overlay]\node[shift={(-1,1)},opacity=0.6] at (current page.south east) {\includegraphics[width=17.5cm]{logo}};
}

\renewcommand{\chaptername}{Lecture}

% \title{Lecture Demonstration}
% \author{An Introduction to Physics through Experiments}
% %\date{5th April, 2018}
% \date{}
% \maketitle

\chapter{Light}

\section{The quality of light}

\subsection{Introduction}

\vspace{0.5cm}

I have always found the study of light to be quite confusing and have only recently been able to properly appreciate why, as well as the richness that the study of light lends itself to. 

For starters, light as a classical wave forms the center of Classical Electromagnetism that you will be studying in some detail in the next semester. 


\begin{tcolorbox}
\paragraph{Question: }Consider the following properties of light:

\begin{enumerate}
\item Wavelength
\item Frequency
\item Speed
\item Energy
\item Polarisation
\end{enumerate}

Which (if any) constants of Nature are associated with them?
\end{tcolorbox}

\vspace{0.5cm}

As a fundamental particle, light is described quantum mechanically as a quantum of energy. The fact that it travels at $c$, the speed limit of the universe, is a reflection of the fact that it is a highly relativistic particle.

Throughout this class we will be jumping between different ways of looking at light, and I feel that this is something you should be comfortable doing yourself, eventually.

\subsection{Colours}

Let us begin with an analysis that is not at all rigorous. The light that you see is a combination of different wavelengths. These forms a very small part of what we call the `electromagnetic spectrum', which ranges all the way from gamma radiation and x-rays to microwaves and radiowaves. The property of colour, however, is only associated with light that falls between the convenient wavelengths of around 400 -- 800 nm, the \textit{visible spectrum}.

Objects appear different colours because they absorb certain wavelength and reflect (or transmit) others. The rods and cones in your eyes detect the intensities and wavelengths of the reflected or transmitted rays and send them to the brain for interpretation. Thus `white' and `black' are not colours in strictest of senses, as they have no specific wavelength of light associated with them. The brain associates an object with the colour white when it reflects all the wavelengths of visible light incident on it, and similarly associates an object with the colour black when it absorbs all (or reflects \textit{none}) of the wavelengths of visible light incident on it. 

For example, a red object looks red because the dye molecules in the paint on it have absorbed the wavelengths of light from the violet/blue end of the spectrum, reflecting only `red' light. If only blue light is shone onto a red shirt, the shirt would appear black, because the blue would be absorbed and there would be no red light to be reflected. When you look at an object, perceiving a `distinct' color, you are not necessarily seeing a single wavelength of light. When you look at an object that appears red to your eye, there may be several frequencies of light striking your eye with varying degrees of intensity. Your eye-brain system, however, interprets the frequencies that strike your eye, interpreting the colour as being `red'.

\begin{tcolorbox}
\paragraph{Question: } Is magenta part of the visible spectrum? If so, at approximately what wavelength is it?
\end{tcolorbox}

\begin{figure}[!htb]
\centering
    \begin{subfigure}[b]{0.3\textwidth}
        \includegraphics[width=\textwidth]{demo_1}
        \caption{A red object in red light}
        \label{img_color_demo_1}
    \end{subfigure}
    ~ %add desired spacing between images, e. g. ~, \quad, \qquad, \hfill etc. 
      %(or a blank line to force the subfigure onto a new line)
    \begin{subfigure}[b]{0.3\textwidth}
        \includegraphics[width=\textwidth]{demo_2}
        \caption{A red object in blue light}
        \label{img_color_demo_2}
    \end{subfigure}
    ~ %add desired spacing between images, e. g. ~, \quad, \qquad, \hfill etc. 
    %(or a blank line to force the subfigure onto a new line)
    \begin{subfigure}[b]{0.3\textwidth}
        \includegraphics[width=\textwidth]{demo_3}
        \caption{A red object in green light}
        \label{img_color_demo_3}
    \end{subfigure}
    
    \begin{subfigure}[b]{0.3\textwidth}
        \includegraphics[width=\textwidth]{demo_4}
        \caption{The addition of colours}
        \label{img_color_demo_3}
    \end{subfigure}
    \caption{Demo on the addition of colours}\label{fig:animals}
\end{figure}

\section{The production of light}

\subsection{Thermal radiation}

Thermal radiation is the conversion of thermal energy into electromagnetic energy. But what exactly does it mean for such a transfer to occur? To understand this properly, we will of course need to understand the structure of the material that composes the matter we are providing the thermal energy to. 

Ultimately, the object producing the light is composed of constituents such as atoms and molecules arranged in some form of a lattice. You should remember from earlier courses that the \textbf{temperature} of an object is a statistical property that -- at microscopic scales -- is a reflection of the average kinetic energy of the constituents of that object. Thus, an increase in temperature would cause these atoms or molecules to start `jiggling' about more erratically (at least on an average) and would thus result in more frequent collisions between them.  

Atoms and molecules have varying behaviours when different energies are imparted to them. For example, the chemical bonds within molecules -- described by Quantum Mechanics -- vibrate at certain specific frequencies. If a collision imparts enough energy to excite one of these vibrational modes of the molecule, the kinetic energy due to the collision serves to transfer the molecule to an excited state. When it then returns back to its original state, it emits a photon having a frequency related to the energy difference between the two states. Similarly, atoms themselves have energy levels, and if a collision excites an electron to a higher level, it too returns to its ground state after emitting a photon.

The vibrational energy levels usually radiate in the infrared regime while the electronic transitions radiate in the ultraviolet. The visible region overlaps these two levels and thus obtains contributions from both. It is important to realise that since the energy is imparted through collisions, there is a continuous range of energies to which atoms and molecules may be excited and thus we get a continuous spectrum of light. However, this radiation has a very specific spectrum. It turns out that objects at the same temperature emitted more or less exactly the same spectrum of light, i.e.\ their intensities of the different wavelengths they emitted were identical. At the beginning of the last century, this was one of the `few' problems that Physics had left to address.

\subsection{Blackbody radiation and the Planck spectrum}

A black body is an idealized physical body that absorbs all incident electromagnetic radiation, regardless of frequency or angle of incidence. At thermal equilibrium (that is, at a constant temperature) it emits electromagnetic radiation called black-body radiation in the manner earlier specified. At room temperature appears it appears black, as most of the energy it radiates is infrared and cannot be perceived by the human eye. When it becomes a little hotter, it appears dull red, and as its temperature increases further it eventually becomes blue-white. The spectrum of a blackbody was known quite well experimentally, however no theoretical approach could describe the entire spectrum accurately.

At the time there were two different laws, each of which fit the experimental data in different regimes, but neither of which explained the full spectrum. For low frequencies (or long wavelengths), the law was the ``Rayleigh-Jeans law'', while for high frequencies (or short wavelengths) it was known as ``Wein's displacement law''.

\begin{figure}[!htb]
\centering
\begin{subfigure}[b]{0.45\textwidth}
        \includegraphics[width=\textwidth]{planck.png}
        \caption{Classical approximations}
        \label{planck_approx}
    \end{subfigure}
    ~ %add desired spacing between images, e. g. ~, \quad, \qquad, \hfill etc. 
      %(or a blank line to force the subfigure onto a new line)
    \begin{subfigure}[b]{0.45\textwidth}
        \hspace{0.5cm}\includegraphics[width=0.8\textwidth]{planck_temp.jpg}
        \caption{Variation with temperature}
        \label{planck_temp}
    \end{subfigure}
\caption{The Planck Spectrum for Blackbodies}
\label{planck}
\end{figure}

The problem was solved by the German physicist Max Planck who realised that `classical' physics could not be used for such blackbody radiation, and thereby ushered in Quantum Mechanics by postulating that the energies were absorbed and emitted in specific quanta. The resulting spectrum was found to be characterised by the temperature of the body and was found to fit a remarkable number of spectra from the sun to the Cosmic Microwave Background Radiation left as a remnant from the Big Bang.

\begin{tcolorbox}
\paragraph{Question: } It is often said that the Cosmic Microwave Background Radiation has a `temperature' of 2.73 K. What do you think this means?
\end{tcolorbox}

\subsection{Vapour lamps and discrete spectra}

As is only to be expected, thermal radiation is not a very efficient way of producing light. The reason for this is that the radiation emitted is not restricted to only the visible spectrum and therefore much of the light that is created cannot be used to illuminate in the general sense of the word. \textbf{Vapour lamps} such as Mercury and Sodium lamps are far more efficient, essentially because they only need enough energy to excite certain electronic spectral lines of the materials within them.

In general, vapour lamps have a low pressure noble gas (Argon in the case of a Mercury lamp and a combination of Neon and Argon for Sodium). A voltage is applied to ionise the gas by creating an electrical arc: the electrons are ripped off the atoms, introducing ions, free electrons and photons into the tube. The heat from the arc vaporises the metal inside the tube. The electrons in the metal atoms then get excited to different energy levels and they cascade down to their initial states by emitting certain special frequencies (wavelengths) of light.

\begin{figure}[!htb]
\centering
\includegraphics[width=0.75\textwidth]{hgspectrum.png}
        \caption{The mercury spectrum}
        \label{hgspectrum}
\end{figure}


\begin{tcolorbox}
\paragraph{Question: } Why do we need Quantum Mechanics to explain the spectrum you see when you look at a Mercury or Sodium vapour lamp? What would it look like if the atom were a classical object? 
\paragraph{Hint: } Consider the atom to be a `solar-system', like the Rutherford model. 

\paragraph{Question: } Can you explain the fact that you \textit{also} see a continuous spectrum on the background of the spectral lines of Sodium or Mercury when you examine their spectra?
\paragraph{Hint: } What causes a continuous spectrum?
\end{tcolorbox}

\subsection{Lasers}

Lasers are devices that emit electromagnetic radiation through a process known as \textbf{stimulated emission}. The term `Laser' itself originated as an acronym of \textit{Light Amplification by Stimulated Emission of Radiation} (which is why spelling the word with a `z' is highly incorrect). We have so far observed the production of light through a process known as \textbf{spontaneous emission} through electronic transitions within an atom. A laser differs from such sources in that it emits light \textit{coherently}, spatially and temporally. 

In the case of spontaneous emission, imagine we have two energy levels $E_0$ and $E_1$. If we manage to excite an electron in an atom to state $E_1$ , it will fall back to $E_0$ spontaneously, emitting an appropriate photon. However, both the direction and the phase of this light will be random. Furthermore, the amount of time the electron spends in this excited state (i.e.\ \textit{when} the photon is emitted) may also vary through the laws of Quantum Mechanics\footnote{Very much like the famous Heisenberg uncertainty principle $\Delta x \Delta p \geq \hbar$, there is another relation $\Delta E \Delta t \geq \hbar$ which roughly states that the time ($\Delta t$) an electron spends in state is inversely proportional to the energy difference between it and the ground state ($\Delta E$).}.

It is, however, desirable to have a coherent, monochromatic source of radiation for many different experiments (some of which you have already performed in the laboratory in these past few sessions). In order to achieve such a source, we will have to turn to another physical process known as stimulated emission.

It turns out that if we manage to excite the electron of an atom (using some sort of `pump') to the energy state $E_1$ and then \textit{pass a photon of exactly the same frequency as $\Delta E/\hbar$} near the atom, there is a very high probability that photon emitted when the $E_1 \rightarrow E_0$ transition occurs will be identical to the passing photon. In this way, we can `amplify' the original passing photon to get two photons having exactly the same phase, direction and wavelength. Such a process is known as `stimulated emission'. 

\begin{figure}[!htb]
\centering
\includegraphics[scale=0.5]{img_emission.png}
\caption{Absorption, Spontaneous Emission and Stimulated Emission}
\label{img_emission}
\end{figure}

\section{The diffraction of light}

Diffraction occurs in waves when they bend around small obstacles or when they spread out after passing through narrow apertures. The standard way of analysing such a problem is some form of `Huygen's Principle', where we define \textit{wavefronts}, each point of which is the source of \textit{secondary waves} which in turn interfere, producing the pattern that you see on the screen. The introduction of these wavefronts -- albeit highly successful -- is slightly ad-hoc and not completely justified physically. This is understandable, as Huygens did not have the right tool (total integral calculus) to describe this phenomenon mathematically. Most of the mathematics was fleshed out by another great Physicist, Augustin-Jean Fresnel. This mathematics is quite involved and as I am certain that you will be subjected to it in great detail during your course on optics, I will not attempt to explain it in too much detail. 

Suffice it to say that the presence of an obstacle of a certain length $a$ in the direction of propagation of light causes some light to shift slightly and traverse a slightly larger distance. This so-called \textit{path-difference} -- dependent on the size of the obstacle $a$ -- induces a difference in \textit{phase} which in turn leads to the light interfering after it passes around the obstacle and producing a pattern of maxima and minima on the screen. Obviously, changing $a$ we would get a different pattern and so a close inspection of the interference pattern provides us with information on the dimension of the obstacle.

\subsection{Babinet's Principle}

Babinet's principle states that

\begin{center}
\textit{Complementary objects produce the same diffraction pattern, except for the intensity of the central maxima.}
\end{center}

We will provide a quick `proof' of this after some mathematical baggage in the subsequent sections. 

\begin{tcolorbox}
\paragraph{Question: }Two objects are complementary if one of them is transparent when the other is opaque and opaque when the other is transparent. What are the complementary objects of the following:

\begin{enumerate}
\item A narrow slit of width $d$.
\item A circular aperture of diameter $d$.
\end{enumerate}
\end{tcolorbox}

Henceforth we will not differentiate between the patterns made by such complementary objects.

\subsection{The single slit}

When the path difference due to a single slit (or thin wire) is exactly equal to an integral number of wavelengths of the light used, the waves \textit{interfere destructively}. As a result, minima occur whenever

\begin{equation*}
\underbrace{a \sin{\theta_n}}_{\text{path difference}} = \underbrace{n \lambda}_{\substack{\text{integral number} \\ \text{of wavelengths}}}
\end{equation*}

I would like to stress that the above way of interpreting mathematical equations is very important, as it attaches a physical significance to every mathematical entity in the equation.


\begin{tcolorbox}
\paragraph{Question: } What happens to $\theta_n$ when you increase $a$? 

\paragraph{Question: } Using the above equation, what is the \textbf{smallest} length you can probe using a light of wavelength $\lambda$? Prove this mathematically.

\paragraph{Question: } Is there a maximum length above which light of a wavelength $\lambda$ can no longer probe? If yes, prove it mathematically. If no, what additional constraints are introduced when probing larger length scales?
\end{tcolorbox}

Observing the pattern in Figure \ref{img_diff} you can clearly see that the intensity is not a constant, but seems to vary with angle. It turns out that -- as a function of the angle $\theta$ -- the intensity falls as 

\begin{equation*}
I(\theta) = I(0) \left( \frac{\sin{\beta}}{\beta}\right)^2 \quad \quad \text{where  } \beta = \frac{\pi a \sin \theta}{\lambda}
\end{equation*}

\begin{figure}[!htb]
\centering
    \begin{subfigure}[b]{0.45\textwidth}
        \includegraphics[width=0.8\textwidth]{fraun_theory.jpg}
     %   \caption{}
        \label{fraun_theory}
    \end{subfigure}
\begin{subfigure}[b]{0.45\textwidth}
       \includegraphics[scale=0.2]{fraun.jpg}
     %   \caption{Pattern on the screen}
        \label{fraun}
    \end{subfigure}
\caption{Diffraction pattern due to a single slit}
\label{img_diff}
\end{figure}


\begin{tcolorbox}
\paragraph{Question: } Can you sketch the above function? Write down an equation to find its zeros. Does it look familiar?
\end{tcolorbox}

\subsection{The single helix}

The single helix has, associated with it, two length scales: the thickness of the wire and the spacing between the wires (the `pitch'). The pitch works wither as two thin wires separated by a distance $P$, or alternatively a single slit of width $P$.

The entire pattern can thus be decomposed into a diffraction grating angled up at an angle $\alpha_1$, a diffraction grating angled down at an angle $\alpha_1$ (both of which have a thickness $a$), and diffraction gratings angled up (and down) at the same angle (of thickness $P$). Thus, these four objects combine together to get the $X$ shaped pattern on the screen.

\subsection{The double helix}

This pattern can be decomposed in a very similar fashion as the single helix, except that there is now another length scale associated with the object: the \textit{separation} between the two helices $d$, which produces another diffraction pattern on each arm of the $X$.



\begin{figure}[!htb]
\centering
\includegraphics[scale=0.2]{helices.png}
\caption{Schematics of a single and double helix}
\label{helix}
\end{figure}

\chapter{The Mathematics of Diffraction}

\section{The Fourier Transform}

It turns out that the diffraction patterns and the obstacles that produce them are very closely related by an extremely cool mathematical relationship called a \textbf{Fourier Transform}. You will be studying about this in great detail in your mathematical physics courses and -- should you wish to continue in Physics --  it will be impossible for you to avoid it in the future. The Fourier Transform, as you might imagine, is quite closely related in principle to the Fourier Series you have already encountered and I will try to motivate this relationship in the subsequent sections.

\subsection{The epicycles}

As I was researching for this class I discovered something absolutely fabulous that I thought I \textit{had} to tell you. To understand this, we will have to go back to the oldest natural science: astronomy. The Greeks -- as most cultures of antiquity -- were very interested in the motions of the heavens. They were also very interested in symmetry and harmony, different ideas which found a synthesis in many of their theories of the world. 

The sphere, of course, being extremely symmetric was their orbit of choice to describe the heavens. Plato (circa 400 BC) came up with the notion of the Earth at the center of the universe and all other objects orbiting it in spherical trajectories. There were, however, certain annoying heavenly bodies that refused to behave as expected. These were called `wanderers', \textit{planétai}. The motion of these planets caused them considerable distress, as they were quite a fly in the ointment of symmetry. A solution to the problem was provided by Ptolemy.

The problem was that if you watch the planets carefully, they sometimes move backwards in the sky. So Ptolemy came up with the following idea: planets move around one big circle, but they move around a little circle at the same time. Imagine spinning a stick around you, the edge of which is attached to a rotating wheel. The planet would move like a point on the edge of the wheel; these circles were known as `epicycles'. This theory turns out to be wrong, but more importantly it is a \textit{bad} theory, for a reason that we shall soon see. It turns out that the orbit of any planet -- viewed from earth -- can be described to arbitrary accuracy by adding enough epicycles, as can be seen \href{https://www.youtube.com/watch?v=QVuU2YCwHjw}{in this amazing YouTube video}.

\begin{figure}[!htb]
\centering
\includegraphics[scale=0.4]{epicycles.png}
\caption{Epicycles and planetary motion}
\label{epicycles}
\end{figure}

The planet moves about in trajectory on a plane. Let us try to understand this mathematically. An easy way to do this in two dimensions is to superimpose a complex plane over real space and consider the motion of this particle as if it were moving about in this plane. You can easily justify this to yourself realising that circles are much easier to represent using complex numbers. The motion of a point on the edge of a single circle rotating at a frequency $\omega$ and having a radius $R$ is given by

\begin{equation*}
z(t) = R e^{i \omega t}
\end{equation*}

For two circles it would be 


\begin{equation*}
z(t) = R_1 e^{i \omega_1 t} + R_2 e^{i \omega_2 t}
\end{equation*}

We could now imagine adding three, four, or even an infinite number of circles, with every possible angular frequency. In this case, the sum would become an \textit{integral}, and the discrete indices of $R_1, R_2, ...$ etc. will be replaced by a continuous function $R(\omega)$. Thus,


\begin{equation*}
z(t) = \int^\infty_{-\infty} R(\omega) e^{i \omega t} \dd \omega
\end{equation*}

This new function $R(\omega)$ is called the Fourier Transform of $z(t)$. Giving you the function $R(\omega)$ is equivalent to giving you all the information in $z(t)$.

Let's examine what this means. It means that any arbitrary time dependent function $z(t)$ can be perfectly emulated by infinitely many circles of different frequencies all added up, provided that their radii are `correctly' weighted with respect to their frequencies (the function $R(\omega)$ we just saw). We could now add a further constraint that would simplify our analysis. Suppose now that the function is \textbf{periodic}, meaning that the path closes back on itself. Then, most frequencies  are no longer necessary, and only integral values of the frequency of the slowest circle contribute to the function. In this case, the integral reverts back to an infinite countable sum.


\begin{equation*}
z(t) = \sum_{n = -\infty}^\infty R_n e^{i n \omega_0 t}
\end{equation*}

which should look familiar. You could take the first ten or twenty and ignore the rest, fitting the data to any level of precision you wish.

\subsection{The inverse transform}

Diffraction, as I already mentioned, produces a pattern on the screen that is a Fourier Transform of the obstacle or aperture. We will use this face to ``prove'' Babinet's Principle, however we still have a little more math to get out of the way. 

We have already seen that if we use the earlier mentioned formula for the Fourier Transform\footnote{The actual definition of the Fourier Transform and its inverse are occasionally reversed, and there are also factors of $2\pi$ that appear and disappear. These, however, do not affect our strictly qualitative analysis.}, then you can also invert the operation exactly as you would in the case of Fourier series. 

\begin{tcolorbox}

\paragraph{Forward Fourier Transform}

\begin{equation*}
z(t) = \int^\infty_{-\infty} R(\omega) e^{i \omega t} \dd \omega
\end{equation*}

\paragraph{Inverse Fourier Transform}

\begin{equation*}
R(\omega) = \int^\infty_{-\infty} z(t) e^{-i \omega t} \dd t
\end{equation*}

\paragraph{Note: } Notice that the two variables $t$ and $\omega$ in question here are related inversely. This is characteristic of Fourier Transforms.

\end{tcolorbox}

\subsection{The Dirac Delta function}

The last mathematical object that we will have to deal with is the Dirac Delta function that most of you must have already heard of. It is a strange function\footnote{Later, you will learn that it is not really even a function, but rather a \textit{distribution}.}. This function is zero everywhere, except at the origin where it is infinite.

\begin{equation*}
\delta (k) = \begin{cases}
              \infty, & k = 0\\
               0,              & \text{otherwise}
             \end{cases}
\end{equation*}

It is defined by its action on other functions. In particular,

\begin{equation*}
\int_{-\infty}^\infty \delta(k) f(k) \dd k = f(0)
\end{equation*}

i.e.\ it picks up \textit{only} the value at 0.

\begin{tcolorbox}
\paragraph{Question: } What is the Fourier Transform of $\delta(k)$?

\paragraph{Question: } What do you think is the Fourier Transform of 1?
\paragraph{Answer: } It turns out to be $\delta(k)$.
\end{tcolorbox}


\subsection{Proving Babinet's Principle}

Here is a rudimentary proof of Babinet's principle. Let us begin by drawing out two complementary objects, a slit and a thin wire of the same thickness.

\begin{figure}[!htb]
\centering
\includegraphics[scale=0.5]{complementary.png}
\caption{Complementary objects: A thin wire and a narrow slit}
\label{img_complementary}
\end{figure}

Let us suppose that we add both of these objects. The resulting pattern will simply be 1, as they are complementary. This will happen to any pair of complementary objects. Thus,

\begin{equation*}
f(x) + f_\text{comp}(x) = 1
\end{equation*}

We know the following three things:

\begin{enumerate}
\item The Fourier Transform is linear (since it's an integral),
\item The Fourier Transform represents what we will see on the screen.
\item The Fourier Transform of 1 is $\delta(k)$.
\end{enumerate}

Thus, if we call $\widetilde{f}(k)$ and $\widetilde{f}_\text{comp}(k)$ the Fourier transforms of $f(x)$ and $f_\text{comp}(x)$, then taking the Fourier transforms of the last equation, we get

\begin{equation*}
\widetilde{f}(k) + \widetilde{f}_\text{comp}(k) = \delta(k)
\end{equation*}

Now this might appear to be a particularly formidable equation to solve, but we shall solve it \textit{à la physicienne}. The Dirac Delta is zero almost everywhere. In fact, it is zero everywhere \textit{except} at the origin (i.e.\ everywhere that Babinet's principle holds!) and so we shall consider every point \textit{except} the point directly in front of the beam\footnote{Remember that Babinet's principle also does not hold for the intensity of the central beam.}. Of course, in this case, $\delta(k) = 0$. Thus,

\begin{equation*}
\widetilde{f}(k) + \widetilde{f}_\text{comp}(k) = 0
\end{equation*}

or in other words,

\begin{equation*}
\widetilde{f}(k) = - \widetilde{f}_\text{comp}(k)
\end{equation*}

which, you will appreciate, is simply a mathematical formulation of Babinet's principle.

\subsection{Examples}

Here are some quick examples, convince yourselves that the mathematical description of the apertures is accurate. The function $\Theta(x)$ is known as Heaviside Step Function. It is defined as

\begin{equation*}
\Theta (x) = \begin{cases}
              0, & x < 0\\
              1, & x > 0
             \end{cases}
\end{equation*}


\begin{figure}[!htb]
\centering
\includegraphics[scale=0.4]{example_1.png}
\caption{A single slit}
\label{example_1}
\end{figure}

\begin{figure}[!htb]
\centering
\includegraphics[scale=0.4]{example_2.png}
\caption{A circular aperture}
\label{example_2}
\end{figure}


\begin{tcolorbox}
\paragraph{Remark: } $J_1(ka)$ is a special function known as a \textbf{Bessel function}. It occurs very frequently in Physics. You may think of it as a ``modified'' sine function, except that instead of at $x=n\pi$, the function has zeros at different values, specified on Figure \ref{example_3} below.

\paragraph{Question: } Can you explain the fact that for circular apertures, the values of $\bar{m}$ are not integers. Can you explain their values?
\end{tcolorbox}

\begin{figure}[!htb]
\centering
\includegraphics[scale=0.4]{example_3.png}
\caption{The Bessel function as a modified sinusoid}
\label{example_3}
\end{figure}

\newpage

\section*{References and additional reading}

\begin{enumerate}
\item \href{https://physics.info/planck/}{Blackbody Radiation - The Physics Hypertextbook}
\item \href{https://www.ifa.hawaii.edu/~barnes/ASTR110L_F05/spectralab.html}{Spectra in the Lab}
\item \href{http://hyperphysics.phy-astr.gsu.edu/hbase/quantum/atspect.html}{Atomic Spectra - Hyperphysics}
\item \href{https://pdfs.semanticscholar.org/8ccd/de212e9059a35c111704073aea2443984614.pdf}{How Rosalind Franklin Discovered the Helical Structure of DNA: Experiments in Diffraction}
\item \href{https://physics.stackexchange.com/questions/383138/diffraction-pattern-due-to-double-helix?rq=1}{Diffraction of a double helix - Physics StackExchange}
\item \href{https://betterexplained.com/articles/an-interactive-guide-to-the-fourier-transform/}{An Interactive Guide To The Fourier Transform}
\item \href{https://nipunbatra.github.io/blog/2016/FT.html}{Dummies guide to Fourier Transform} (This involves programming using Python)
\item \href{https://math.stackexchange.com/questions/1002/fourier-transform-for-dummies}{Fourier Transform for Dummies - Mathematics StackExchange}
\item \href{https://www.youtube.com/watch?v=QVuU2YCwHjw}{Ptolemy and Homer (Simpson) - YouTube}
\item \href{http://brettcvz.github.io/epicycles/}{The Epicycle Demonstrator - Drawing by Epicycles} (This site has some wacky examples, including the letter 'B')
\item \href{http://www.cchem.berkeley.edu/chem120a/extra/delta_functions.pdf}{Delta Functions}
\end{enumerate}


\newpage


\vbox{
\textcolor{Blue}{\part{Assessment}}
\tikz[remember picture,overlay]\node[shift={(-1,1)},opacity=0.6] at (current page.south east) {\includegraphics[width=17.5cm]{logo}};
}

\renewcommand{\chaptername}{Exam}

\input{11_Written_Exam.tex}

\end{document}